\usepackage[T1]{fontenc}
\usepackage{lmodern}
\usepackage{xspace}
\usepackage{graphicx}
\usepackage{slatex}
\usepackage{subfig}
\usepackage{upgreek}
\usepackage{varwidth}
\usepackage{verbatim}
\usepackage{color}
\usepackage{comment}
\usepackage{amsmath, amssymb}
\usepackage{stmaryrd}
\usepackage{amscd}
\usepackage{hyperref}
\usepackage{mathtools}
%\usepackage{xcolor}

%\hypersetup{
%	colorlinks,
%	linkcolor={red!50!black},
%	citecolor={blue!50!black},
%	urlcolor={blue!80!black}
%}

%%%%%%%%%%%%%%%%%%%%%%%%%%%%%%%%%%%%%%%%%%%%%%%%%%%%%%%%%%%%%%%%%%%%%%%%%%%%%%%%%%%%%%%%%%%%%%%%%%%%%
% SPACE HACKS
%%%%%%%%%%%%%%%%%%%%%%%%%%%%%%%%%%%%%%%%%%%%%%%%%%%%%%%%%%%%%%%%%%%%%%%%%%%%%%%%%%%%%%%%%%%%%%%%%%%%%

% Uncomment the following two lines to get a good idea how much vertical space is being wasted
%\raggedbottom
%\usepackage{showframe}

\makeatletter

\renewcommand\normalsize{%
   \@setfontsize\normalsize\@xpt{12pt}%
   \abovedisplayskip 6.5\p@ \@minus4\p@
   \belowdisplayskip 6.5\p@ \@minus4\p@
   \abovedisplayshortskip \z@
   \belowdisplayshortskip 4\p@ \@minus2\p@
   \belowdisplayskip \abovedisplayskip
}

\renewcommand\small{%
   \@setfontsize\small\@ixpt{11}%
   \abovedisplayskip 0\p@
   \belowdisplayskip 0\p@
   \abovedisplayshortskip 0\p@
   \belowdisplayshortskip 0\p@
   \belowdisplayskip \abovedisplayskip
}

\setlength\floatsep       {3pt \@minus 1\p@}
\setlength\textfloatsep   {12pt \@minus 4\p@}
\setlength\intextsep      {3pt \@minus 1\p@}
\setlength\dblfloatsep    {3pt \@minus 1\p@}
\setlength\dbltextfloatsep{3pt \@minus 1\p@}
\makeatother

\usepackage{enumitem}
\setlist{noitemsep,topsep=2pt}
\setitemize{noitemsep,topsep=2pt}
\setenumerate{noitemsep,topsep=2pt}

%\widowpenalty=14
%\clubpenalty=14

%%%%%%%%%%%%%%%%%%%%%%%%%%%%%%%%%%%%%%%%%%%%%%%%%%%%%%%%%%%%%%%%%%%%%%%%%%%%%%%%%%%%%%%%%%%%%%%%%%%%%
% END SPACE HACKS
%%%%%%%%%%%%%%%%%%%%%%%%%%%%%%%%%%%%%%%%%%%%%%%%%%%%%%%%%%%%%%%%%%%%%%%%%%%%%%%%%%%%%%%%%%%%%%%%%%%%%

\newcommand\numberthis{\addtocounter{equation}{1}\tag{\theequation}}

% puts a little space between the \hrule and captions
%\belowcaptionskip 0.25\baselineskip
\newcommand{\bottomhrule}{\vspace{3pt}\hrule\vspace{3pt}}

% keywords
\newcommand{\keyword}[1]{\textbf{#1}}
\newcommand{\mykeyword}[1]{\textbf{#1}}

\newcommand\xqed[1]{%
  \leavevmode\unskip\penalty9999 \hbox{}\nobreak\hfill
  \quad\hbox{#1}}

\newcommand{\exampleqed}{\xqed{$\diamondsuit$}}


\newcommand{\lzfclang}{\ensuremath{\lambda_{\text{ZFC}}}\xspace}

\newcommand{\uscore}{\underbar{\ \,}}

%
% Language phase 0: meta
%
% Mostly semantic functions and grammars, definitional extensions to first-order logic

\newcommand{\metadef}{:=}
\newcommand{\metastx}{:\equiv}
\newcommand{\objdef}{:=}
\newcommand{\objstx}{:\equiv}
\newcommand{\gor}{\ |\ }
\newcommand{\meaningof}[1]{\left\llbracket{#1}\right\rrbracket}
\newcommand{\enc}[1]{\mathcal{F}\!\meaningof{#1}}
\newcommand{\stx}[1]{\mathcal{S}\!\meaningof{#1}}
%\newcommand{\metasubst}[3]{{#1}[{#2}:={#3}]}
\newcommand{\metasubst}[3]{s\!\meaningof{{#1},{#2},{#3}}}
\newcommand{\imp}{\Rightarrow}
\newcommand{\rimp}{\Leftarrow}
\DeclareMathOperator{\disjoint}{\mbox{$\not\mspace{-5mu}\i$}}

\renewcommand{\dots}{...}
\newcommand{\setb}[2]{\lbrace {#1} \ \lvert\ {#2} \rbrace}

\newcommand{\justdenom}[1]{\mspace{-4mu} {} \atop {#1} \mspace{-4mu}}
\newcommand{\djustdenom}[1]{\displaystyle{\justdenom{#1}}}


%
% Language phase 1: first-order logic
%

% Common notation and names

\newcommand{\set}[1]{\{{#1}\}}
\newcommand{\seq}[1]{\left({#1}\right)}
\newcommand{\pair}[1]{\langle{#1}\rangle}
\newcommand{\dpair}[1]{\left\langle{#1}\right\rangle}

\newcommand{\band}{\wedge}
\newcommand{\pand}{\big\wedge}
\newcommand{\bor}{\vee}
\newcommand{\por}{\big\vee}

\newcommand{\Bool}{\mathbb{B}}
\newcommand{\Nat}{\mathbb{N}}
\newcommand{\Int}{\mathbb{Z}}
\newcommand{\Rat}{\mathbb{Q}}
\renewcommand{\Re}{\mathbb{R}}
\renewcommand{\P}{\mathbb{P}}

\newcommand{\U}{\textstyle\bigcup}
\renewcommand{\u}{\cup}
\newcommand{\I}{\textstyle\bigcap}
\renewcommand{\i}{\cap}
\newcommand{\wo}{\backslash}
\let\oldemptyset\emptyset
\renewcommand{\emptyset}{\varnothing}
\newcommand{\powerset}{\mathcal{P}}

\newcommand{\Forall}[1]{\forall\,{#1}\ldotp\,}
\newcommand{\Exists}[1]{\exists\,{#1}\ldotp\,}
\newcommand{\ExistsOne}[1]{\exists!\,{#1}\ldotp\,}
\newcommand{\The}[1]{\iota\,{#1}\ldotp\,}

\newcommand{\tto}{\Rightarrow}
%\newcommand{\bijto}{\xrightarrow{\mathrm{bij}}}
\newcommand{\bijto}{\leftrightarrow}
\newcommand{\Ord}{\mathit{Ord}}

\newcommand{\A}{\mathcal{A}}
\newcommand{\B}{\mathcal{B}}
\newcommand{\C}{\mathcal{C}}
\newcommand{\E}{\mathcal{E}}
\newcommand{\V}{\mathcal{V}}

\newcommand{\pnand}{\mathit{nand}}
\newcommand{\w}{\backslash}

% Invented notation and names

\newcommand{\myfun}[1]{\mathit{#1}}
\newcommand{\encset}{\myfun{set}}
\newcommand{\lex}{\myfun{lex}}
\newcommand{\addset}{\myfun{put}}

%\newcommand{\dto}{\Downarrow}
% makes the horizontal spacing around the arrow nice:
\DeclareMathOperator{\dto}{\Downarrow}

\newcommand{\plusomega}{+_\lnat}
\newcommand{\timesomega}{\times_\lnat}

\newcommand{\plusint}{+_Z}
\newcommand{\minusint}{-_Z}
\newcommand{\timesint}{\times_Z}
\newcommand{\equalint}{=_{\mathsf{Z}}}

\newcommand{\plusz}{+_\Int}
\newcommand{\minusz}{-_\Int}
\newcommand{\timesz}{\times_\Int}

\newcommand{\equalrat}{=_{\mathsf{Q}}}
\newcommand{\plusq}{+_\Rat}
\newcommand{\minusq}{-_\Rat}
\newcommand{\timesq}{\times_\Rat}
\newcommand{\divq}{\div_\Rat}

%
% Language phase 2: lambda_ZFC
%

\newcommand{\objlang}[1]{\mathsf{#1}}
%\newcommand{\objlang}[1]{\mathtt{#1}}

\renewcommand{\choose}{\mathcal{E}}
%\newcommand{\Choose}[1]{\choose\,{#1}\ldotp\,}
\newcommand{\Choose}[1]{\set{#1}\ldotp\,}

\newcommand{\domain}{\objlang{domain}}
\newcommand{\range}{\objlang{range}}
\newcommand{\image}{\objlang{image}}
\newcommand{\filter}{\objlang{select}}

\newcommand{\lunion}[1]{{\textstyle\U}\ {#1}}
\newcommand{\lpowerset}[1]{\powerset\ {#1}}
\newcommand{\limage}[2]{{\objlang{image}}\ {#1}\ {#2}}
\newcommand{\lorder}[1]{{\objlang{card}}\ {#1}}
\newcommand{\ltake}[1]{{\objlang{take}}\ {#1}}
\newcommand{\lemptyset}{\oldemptyset}
\newcommand{\lnat}{\upomega}
\newcommand{\lif}[3]{{\objlang{if}}\ {#1}\ {#2}\ {#3}}
\newcommand{\ltrue}{\objlang{true}}
\newcommand{\lfalse}{\objlang{false}}
\newcommand{\lfun}{\lambda\ldotp\,}

\newcommand{\fun}[1]{\lambda\mspace{2mu}{#1}\ldotp\,}

\newcommand{\tvar}{t_\mathrm{var}}
\newcommand{\tapp}{t_\mathrm{app}}
\newcommand{\tif}{t_\mathrm{if}}
\newcommand{\tin}{t_\in}
\newcommand{\tunion}{t_\u}
\newcommand{\ttake}{t_\mathrm{take}}
\newcommand{\tpowerset}{t_\powerset}
\newcommand{\timage}{t_\mathrm{image}}
\newcommand{\torder}{t_\mathrm{card}}
\newcommand{\tset}{t_\mathrm{set}}
\newcommand{\tatom}{t_\mathrm{atom}}
\newcommand{\tfun}{t_\lambda}
\newcommand{\atrue}{a_\mathrm{true}}
\newcommand{\afalse}{a_\mathrm{false}}
\newcommand{\ttrue}{t_\mathrm{true}}
\newcommand{\tfalse}{t_\mathrm{false}}

\newcommand{\Vatom}{V_\mathrm{atom}}
\newcommand{\Vset}{V_\mathrm{set}}
\newcommand{\Vfun}{V_\lambda}
\newcommand{\Evar}{E_\mathrm{var}}
\newcommand{\Eapp}{E_\mathrm{app}}
\newcommand{\Eset}{E_\mathrm{set}}
\newcommand{\Eif}{E_\mathrm{if}}
\newcommand{\Ein}{E_\in}
\newcommand{\Eunion}{E_\u}
\newcommand{\Epowerset}{E_\powerset}
\newcommand{\Eimage}{E_\mathrm{image}}
\newcommand{\Eorder}{E_\mathrm{card}}

\newcommand{\fst}{\myfun{fst}}
\newcommand{\snd}{\myfun{snd}}

\newcommand{\jand}{\;\;\,}

\def\mycodehook{\ttcodehook}
\let\schemecodehook\mycodehook

\DeclareMathVersion{sans}
\SetSymbolFont{operators}{sans}{T1}{\sfdefault}{m}{n}
\SetSymbolFont{letters}{sans}{T1}{\sfdefault}{m}{n}
\SetMathAlphabet\mathbf{sans}{T1}{\sfdefault}{bx}{n}
\SetMathAlphabet\mathsf{sans}{T1}{\sfdefault}{m}{n}
\SetMathAlphabet\mathit{sans}{T1}{\rmdefault}{m}{it}

\DeclareSymbolFont{greekletters}{OML}{cmss}{m}{n}
\DeclareMathSymbol{\Gamma}  {\mathord}{greekletters}{"00}
\DeclareMathSymbol{\Delta}  {\mathord}{greekletters}{"01}
\DeclareMathSymbol{\Theta}  {\mathord}{greekletters}{"02}
\DeclareMathSymbol{\Lambda} {\mathord}{greekletters}{"03}
\DeclareMathSymbol{\Xi}     {\mathord}{greekletters}{"04}
\DeclareMathSymbol{\Pi}     {\mathord}{greekletters}{"05}
\DeclareMathSymbol{\Sigma}  {\mathord}{greekletters}{"06}
\DeclareMathSymbol{\Upsilon}{\mathord}{greekletters}{"07}
\DeclareMathSymbol{\Phi}    {\mathord}{greekletters}{"08}
\DeclareMathSymbol{\Psi}    {\mathord}{greekletters}{"09}
\DeclareMathSymbol{\Omega}  {\mathord}{greekletters}{"0A}

\DeclareMathSymbol{\alpha}{\mathord}{greekletters}{"0B}
\DeclareMathSymbol{\beta}{\mathord}{greekletters}{"0C}
\DeclareMathSymbol{\gamma}{\mathord}{greekletters}{"0D}
\DeclareMathSymbol{\delta}{\mathord}{greekletters}{"0E}
\DeclareMathSymbol{\epsilon}{\mathord}{greekletters}{"0F}
\DeclareMathSymbol{\zeta}{\mathord}{greekletters}{"10}
\DeclareMathSymbol{\eta}{\mathord}{greekletters}{"11}
\DeclareMathSymbol{\theta}{\mathord}{greekletters}{"12}
\DeclareMathSymbol{\iota}{\mathord}{greekletters}{"13}
\DeclareMathSymbol{\kappa}{\mathord}{greekletters}{"14}
\DeclareMathSymbol{\lambda}{\mathord}{greekletters}{"15}
\DeclareMathSymbol{\mu}{\mathord}{greekletters}{"16}
\DeclareMathSymbol{\nu}{\mathord}{greekletters}{"17}
\DeclareMathSymbol{\xi}{\mathord}{greekletters}{"18}
\DeclareMathSymbol{\pi}{\mathord}{greekletters}{"19}
\DeclareMathSymbol{\rho}{\mathord}{greekletters}{"1A}
\DeclareMathSymbol{\sigma}{\mathord}{greekletters}{"1B}
\DeclareMathSymbol{\tau}{\mathord}{greekletters}{"1C}
\DeclareMathSymbol{\upsilon}{\mathord}{greekletters}{"1D}
\DeclareMathSymbol{\phi}{\mathord}{greekletters}{"1E}
\DeclareMathSymbol{\chi}{\mathord}{greekletters}{"1F}
\DeclareMathSymbol{\psi}{\mathord}{greekletters}{"20}
\DeclareMathSymbol{\omega}{\mathord}{greekletters}{"21}
\DeclareMathSymbol{\varepsilon}{\mathord}{greekletters}{"22}
\DeclareMathSymbol{\vartheta}{\mathord}{greekletters}{"23}
\DeclareMathSymbol{\varpi}{\mathord}{greekletters}{"24}
\DeclareMathSymbol{\varrho}{\mathord}{greekletters}{"25}
\DeclareMathSymbol{\varsigma}{\mathord}{greekletters}{"26}

\DeclareSymbolFont{fixpunct}{T1}{\rmdefault}{m}{n}
\DeclareSymbolFont{fixpunctoml}{OML}{\rmdefault}{m}{n}
\DeclareMathSymbol{,}{\mathpunct}{fixpunct}{`,}
\DeclareMathSymbol{.}{\mathord}{fixpunct}{`.}
\DeclareMathSymbol{\ldotp}{\mathord}{fixpunct}{`.}
\DeclareMathSymbol{!}{\mathord}{letters}{`-}
\DeclareMathSymbol{/}{\mathpunct}{fixpunct}{`/}
\DeclareMathSymbol{\rightharpoonup}{\mathrel}{fixpunctoml}{"2A}

\DeclareSymbolFont{numbers}{T1}{pvh}{m}{n}
\SetSymbolFont{numbers}{sans}{T1}{\rmdefault}{m}{n}
\DeclareMathSymbol{0}{\mathalpha}{numbers}{"30}
\DeclareMathSymbol{1}{\mathalpha}{numbers}{"31}
\DeclareMathSymbol{2}{\mathalpha}{numbers}{"32}
\DeclareMathSymbol{3}{\mathalpha}{numbers}{"33}
\DeclareMathSymbol{4}{\mathalpha}{numbers}{"34}
\DeclareMathSymbol{5}{\mathalpha}{numbers}{"35}
\DeclareMathSymbol{6}{\mathalpha}{numbers}{"36}
\DeclareMathSymbol{7}{\mathalpha}{numbers}{"37}
\DeclareMathSymbol{8}{\mathalpha}{numbers}{"38}
\DeclareMathSymbol{9}{\mathalpha}{numbers}{"39}

\renewcommand{\notin}{\not\in}

\newcommand\mathtest[2]{\mathchoice{#1}{#2}{#2}{#2}}

\newenvironment{lzfcenv}[1][l]%
{%
	\begin{array}[t]{@{}#1@{}}%
}%
{%
	\end{array}%
}

\newcommand{\tab}{\ \ \ }

\newcommand{\lzfc}[2][l]{\begin{lzfcenv}[#1]#2\end{lzfcenv}}
\newcommand{\tlzfc}[1]{\ensuremath{#1}}

\newcommand{\lzfcsplit}[2][@{}]{\lzfc[r #1 l]{#2}}

\newcommand{\lzfclet}[2]{\lzfc{let\ \ \lzfc[r@{\ \objdef\ }l]{#1} \\ \lzfc{in\ \ #2}}}
\newcommand{\lzfccond}[2][\ \longrightarrow\ ]{\lzfc{cond\ \ \lzfc[l@{\ {#1}\ }l]{#2}}}
\newcommand{\lzfccase}[3][\ \longrightarrow\ ]{\lzfc{case\ \ {#2}\ \ of\\\ \ \lzfc[l@{\ {#1}\ }l]{#3}}}
\newcommand{\lzfcif}[3]{\lzfcsplit{&if~{#1}\\&~~then~{#2}\\&~~else~{#3}}}

\newcommand{\dash}{\text{-}}

%
% Racket syntax (must be last - or at least \defschememathescape must be - see last line)
%

% DrRacket's default colors
\definecolor{identifiercolor}{rgb}{0.15,0.15,0.5}
\definecolor{keywordcolor}{rgb}{0.0,0.0,0.0}
\definecolor{constantcolor}{rgb}{0.16,0.5,0.15}
\definecolor{parenthesiscolor}{rgb}{0.52,0.24,0.14}

\def\keywordfont#1{\textcolor{keywordcolor}{#1}}
\def\variablefont#1{\textcolor{identifiercolor}{#1}}
\def\constantfont#1{\textcolor{constantcolor}{#1}}
\def\datafont#1{\textcolor{parenthesiscolor}{#1}}

\makeatletter
\def\mycodehook{\tt\color{parenthesiscolor}\@setfontsize\small{9.0pt}{10.0pt minus0.5pt}}
\makeatother
\let\schemecodehook\mycodehook
\setlength{\leftcodeskip}{\parindent}

\setlength{\leftcodeskip}{0pt}
\setlength{\rightcodeskip}{0pt plus 1fil}
\setlength{\abovecodeskip}{0.5\abovedisplayskip}
\setlength{\belowcodeskip}{0.5\belowdisplayskip}

\unsetspecialsymbol{.}
\unsetspecialsymbol{...}
\unsetspecialsymbol{-}
\unsetspecialsymbol{1-}
\unsetspecialsymbol{-1+}

\setspecialsymbol{<blank-line>}{\va{}}
\setspecialsymbol{lambda}{\va{$\mathtt{\lambda}$}}
\setspecialsymbol{λ}{\va{$\mathtt{\lambda}$}}
\setspecialsymbol{~}{\va{$\mathtt{\sim}$}}
\setspecialsymbol{_}{\va{$\underbar{ }$}}
\setspecialsymbol{Omega}{\va{$\mathtt{\Omega}$}}
\setspecialsymbol{omega}{\va{$\mathtt{\omega}$}}
\setspecialsymbol{Ω}{\va{$\mathtt{\Omega}$}}
\setspecialsymbol{ω}{\va{$\mathtt{\omega}$}}

\setkeyword{define/drbayes struct/drbayes}

\defschememathescape{$}
