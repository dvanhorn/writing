%-----------------------------------------------------------------------------
%
%               Template for sigplanconf LaTeX Class
%
% Name:         sigplanconf-template.tex
%
% Purpose:      A template for sigplanconf.cls, which is a LaTeX 2e class
%               file for SIGPLAN conference proceedings.
%
% Guide:        Refer to "Author's Guide to the ACM SIGPLAN Class,"
%               sigplanconf-guide.pdf
%
% Author:       Paul C. Anagnostopoulos
%               Windfall Software
%               978 371-2316
%               paul@windfall.com
%
% Created:      15 February 2005
%
%-----------------------------------------------------------------------------

\documentclass[preprint]{sigplanconf}

\usepackage{lmodern}
\usepackage[T1]{fontenc}
\usepackage{xspace}
\usepackage{graphicx}
\usepackage{slatex}
\usepackage{subfig}
\usepackage{upgreek}
\usepackage{varwidth}
\usepackage{verbatim}
\usepackage{color}
\usepackage{stmaryrd}
\usepackage{amsmath}
\usepackage{amssymb}
\usepackage{amsthm}
\usepackage{amscd}
\usepackage{mathtools}

\usepackage{xparse}
\makeatletter
\NewDocumentCommand{\raisedminus}{m}{%
  \raisebox{0.2em}{$\m@th#1{-}$}%
}
\NewDocumentCommand{\unaryminus}{}{%
  \mathbin{%
    \mathchoice{%
      \raisedminus\scriptstyle
    }{%
      \raisedminus\scriptstyle
    }{%
      \raisedminus\scriptscriptstyle
    }{%
      \raisedminus\scriptscriptstyle
    }%
  }%
}
\makeatother

% Hyperref likes to go last
\usepackage[
    bookmarks=true,
    bookmarksnumbered=true,
    breaklinks=false,
    raiselinks=true,
    pdfborder={0 0 0},
    colorlinks=false,
    plainpages=false,
    ]{hyperref}

% This makes hyperlinks point to the tops of figures, not their captions
\usepackage[all]{hypcap}

%%%%%%%%%%%%%%%%%%%%%%%%%%%%%%%%%%%%%%%%%%%%%%%%%%%%%%%%%%%%%%%%%%%%%%%%%%%%%%%%%%%%%%%%%%%%%%%%%%%%%
% BEGIN SPACE HACKS
%%%%%%%%%%%%%%%%%%%%%%%%%%%%%%%%%%%%%%%%%%%%%%%%%%%%%%%%%%%%%%%%%%%%%%%%%%%%%%%%%%%%%%%%%%%%%%%%%%%%%

%\usepackage{showframe}

\makeatletter
\renewcommand\normalsize{%
   \@setfontsize\normalsize\@@ptsize{14.5pt}%
   \abovedisplayskip 8\p@ \@plus2\p@ \@minus2\p@
   \belowdisplayskip \abovedisplayskip
   \abovedisplayshortskip 4\p@ \@plus2\p@ \@minus2\p@
   \belowdisplayshortskip \abovedisplayshortskip
   \renewcommand\arraystretch{0.65}
}

\renewcommand\small{%
   \@setfontsize\small\@ixpt{12pt}%
   \abovedisplayskip 7\p@ \@plus2\p@ \@minus2\p@
   \belowdisplayskip \abovedisplayskip
   %\abovedisplayshortskip 0\p@
   %\belowdisplayshortskip 0\p@
   %\belowdisplayskip \abovedisplayskip
   \renewcommand\arraystretch{0.9}
}

%\setlength\floatsep       {3pt \@minus 1\p@}
%\setlength\textfloatsep   {12pt \@minus 4\p@}
%\setlength\intextsep      {3pt \@minus 1\p@}
%\setlength\dblfloatsep    {3pt \@minus 1\p@}
%\setlength\dbltextfloatsep{3pt \@minus 1\p@}
\makeatother

\usepackage{enumitem}
\setlist{noitemsep,topsep=2pt}
\setitemize{noitemsep,topsep=2pt}
\setenumerate{noitemsep,topsep=2pt}

\newcommand{\smallmathfont}{\small}

%%%%%%%%%%%%%%%%%%%%%%%%%%%%%%%%%%%%%%%%%%%%%%%%%%%%%%%%%%%%%%%%%%%%%%%%%%%%%%%%%%%%%%%%%%%%%%%%%%%%%
% END SPACE HACKS
%%%%%%%%%%%%%%%%%%%%%%%%%%%%%%%%%%%%%%%%%%%%%%%%%%%%%%%%%%%%%%%%%%%%%%%%%%%%%%%%%%%%%%%%%%%%%%%%%%%%%

% Hide subsections in TOC
\setcounter{tocdepth}{1}

\newcounter{maththing}\numberwithin{section}{chapter}

\theoremstyle{definition}
\newtheorem{example}{Example}[chapter]

\theoremstyle{plain}
\newtheorem{definition}[example]{Definition}
\newtheorem{theorem}[example]{Theorem}
\newtheorem{lemma}[example]{Lemma}
\newtheorem{corollary}[example]{Corollary}

\newcommand\numberthis{\addtocounter{equation}{1}\tag{\theequation}}

% puts a little space between the \hrule and captions
\belowcaptionskip 0.25\baselineskip
\newcommand{\bottomhrule}{\vspace{3pt}\hrule\vspace{3pt}}

% keywords
\newcommand{\keyword}[1]{\textbf{#1}}
\newcommand{\mykeyword}[1]{\textbf{\textit{#1}}}

\newcommand\xqed[1]{%
  \leavevmode\unskip\penalty9999 \hbox{}\nobreak\hfill
  \quad\hbox{#1}}

\newcommand{\exampleqed}{\xqed{$\diamondsuit$}}


\newcommand{\lzfclang}{\ensuremath{\lambda_{\text{ZFC}}}\xspace}

%
% Language phase 0: meta
%
% Mostly semantic functions and grammars, definitional extensions to first-order logic

\newcommand{\metadef}{:=}
\newcommand{\metastx}{:\equiv}
\newcommand{\objdef}{:=}
\newcommand{\objstx}{:\equiv}
\newcommand{\gor}{\ |\ }
\newcommand{\meaningof}[1]{\left\llbracket{#1}\right\rrbracket}
\newcommand{\enc}[1]{\mathcal{F}\!\meaningof{#1}}
\newcommand{\stx}[1]{\mathcal{S}\!\meaningof{#1}}
%\newcommand{\metasubst}[3]{{#1}[{#2}:={#3}]}
\newcommand{\metasubst}[3]{s\!\meaningof{{#1},{#2},{#3}}}
\newcommand{\imp}{\Rightarrow}
\newcommand{\rimp}{\Leftarrow}
\DeclareMathOperator{\disjoint}{\mbox{$\not\mspace{-5mu}\i$}}

\renewcommand{\dots}{...}
\newcommand{\setb}[2]{\lbrace {#1} \ \lvert\ {#2} \rbrace}

\newcommand{\justdenom}[1]{\mspace{-4mu} {} \atop {#1} \mspace{-4mu}}
\newcommand{\djustdenom}[1]{\displaystyle{\justdenom{#1}}}


%
% Language phase 1: first-order logic
%

% Common notation and names

\newcommand{\set}[1]{\{{#1}\}}
\newcommand{\seq}[1]{\left({#1}\right)}
\newcommand{\pair}[1]{\langle{#1}\rangle}
\newcommand{\dpair}[1]{\left\langle{#1}\right\rangle}

\newcommand{\band}{\wedge}
\newcommand{\pand}{\big\wedge}
\newcommand{\bor}{\vee}
\newcommand{\por}{\big\vee}

\newcommand{\Bool}{\mathbb{B}}
\newcommand{\Nat}{\mathbb{N}}
\newcommand{\Int}{\mathbb{Z}}
\newcommand{\Rat}{\mathbb{Q}}
\renewcommand{\Re}{\mathbb{R}}
\renewcommand{\P}{\mathbb{P}}

\newcommand{\U}{\textstyle\bigcup}
\renewcommand{\u}{\cup}
\newcommand{\I}{\textstyle\bigcap}
\renewcommand{\i}{\cap}
\newcommand{\wo}{\backslash}
\let\oldemptyset\emptyset
\renewcommand{\emptyset}{\varnothing}
\newcommand{\powerset}{\mathcal{P}}

\newcommand{\Forall}[1]{\forall\,{#1}\ldotp\,}
\newcommand{\Exists}[1]{\exists\,{#1}\ldotp\,}
\newcommand{\ExistsOne}[1]{\exists\char`^\,{#1}\ldotp\,}
\newcommand{\The}[1]{\iota\,{#1}\ldotp\,}

\newcommand{\tto}{\Rightarrow}
%\newcommand{\bijto}{\xrightarrow{\mathrm{bij}}}
\newcommand{\bijto}{\leftrightarrow}
\newcommand{\Ord}{\mathit{Ord}}

\newcommand{\A}{\mathcal{A}}
\newcommand{\B}{\mathcal{B}}
\newcommand{\C}{\mathcal{C}}
\newcommand{\E}{\mathcal{E}}
\newcommand{\V}{\mathcal{V}}

\newcommand{\pnand}{\mathit{nand}}
\newcommand{\w}{\backslash}

% Invented notation and names

\newcommand{\myfun}[1]{\mathit{#1}}
\newcommand{\encset}{\myfun{set}}
\newcommand{\lex}{\myfun{lex}}
\newcommand{\addset}{\myfun{put}}

%\newcommand{\dto}{\Downarrow}
% makes the horizontal spacing around the arrow nice:
\DeclareMathOperator{\dto}{\Downarrow}

\newcommand{\plusomega}{+_\lnat}
\newcommand{\timesomega}{\times_\lnat}

\newcommand{\plusint}{+_Z}
\newcommand{\minusint}{-_Z}
\newcommand{\timesint}{\times_Z}
\newcommand{\equalint}{=_{\mathsf{Z}}}

\newcommand{\plusz}{+_\Int}
\newcommand{\minusz}{-_\Int}
\newcommand{\timesz}{\times_\Int}

\newcommand{\equalrat}{=_{\mathsf{Q}}}
\newcommand{\plusq}{+_\Rat}
\newcommand{\minusq}{-_\Rat}
\newcommand{\timesq}{\times_\Rat}
\newcommand{\divq}{\div_\Rat}

%
% Language phase 2: lambda_ZFC
%

\newcommand{\objlang}[1]{\mathsf{#1}}
%\newcommand{\objlang}[1]{\mathtt{#1}}

\renewcommand{\choose}{\mathcal{E}}
%\newcommand{\Choose}[1]{\choose\,{#1}\ldotp\,}
\newcommand{\Choose}[1]{\set{#1}\ldotp\,}

\newcommand{\domain}{\objlang{domain}}
\newcommand{\range}{\objlang{range}}
\newcommand{\image}{\objlang{image}}
\newcommand{\filter}{\objlang{select}}

\newcommand{\lunion}[1]{{\textstyle\U}\ {#1}}
\newcommand{\lpowerset}[1]{\powerset\ {#1}}
\newcommand{\limage}[2]{{\objlang{image}}\ {#1}\ {#2}}
\newcommand{\lorder}[1]{{\objlang{card}}\ {#1}}
\newcommand{\ltake}[1]{{\objlang{take}}\ {#1}}
\newcommand{\lemptyset}{\oldemptyset}
\newcommand{\lnat}{\upomega}
\newcommand{\lif}[3]{{\objlang{if}}\ {#1}\ {#2}\ {#3}}
\newcommand{\ltrue}{\objlang{true}}
\newcommand{\lfalse}{\objlang{false}}
\newcommand{\lfun}{\lambda\ldotp\,}

\newcommand{\fun}[1]{\lambda\mspace{2mu}{#1}\ldotp\,}

\newcommand{\tvar}{t_\mathrm{var}}
\newcommand{\tapp}{t_\mathrm{app}}
\newcommand{\tif}{t_\mathrm{if}}
\newcommand{\tin}{t_\in}
\newcommand{\tunion}{t_\u}
\newcommand{\ttake}{t_\mathrm{take}}
\newcommand{\tpowerset}{t_\powerset}
\newcommand{\timage}{t_\mathrm{image}}
\newcommand{\torder}{t_\mathrm{card}}
\newcommand{\tset}{t_\mathrm{set}}
\newcommand{\tatom}{t_\mathrm{atom}}
\newcommand{\tfun}{t_\lambda}
\newcommand{\atrue}{a_\mathrm{true}}
\newcommand{\afalse}{a_\mathrm{false}}
\newcommand{\ttrue}{t_\mathrm{true}}
\newcommand{\tfalse}{t_\mathrm{false}}

\newcommand{\Vatom}{V_\mathrm{atom}}
\newcommand{\Vset}{V_\mathrm{set}}
\newcommand{\Vfun}{V_\lambda}
\newcommand{\Evar}{E_\mathrm{var}}
\newcommand{\Eapp}{E_\mathrm{app}}
\newcommand{\Eset}{E_\mathrm{set}}
\newcommand{\Eif}{E_\mathrm{if}}
\newcommand{\Ein}{E_\in}
\newcommand{\Eunion}{E_\u}
\newcommand{\Epowerset}{E_\powerset}
\newcommand{\Eimage}{E_\mathrm{image}}
\newcommand{\Eorder}{E_\mathrm{card}}

\newcommand{\fst}{\myfun{fst}}
\newcommand{\snd}{\myfun{snd}}

\newcommand{\jand}{\;\;\,}

\newenvironment{displaybreaks}%
{%
	\begingroup%
	\allowdisplaybreaks%
}%
{%
	\endgroup%
	\ignorespacesafterend%
}

%\excludecomment{proof}

\newcommand{\figref}[1]{Fig.~\ref{#1}}
\newcommand{\figsref}[1]{Figs.~\ref{#1}}

\newcommand{\arrow}{\rightsquigarrow}

\newcommand{\restrict}[1]{\lvert_{#1}}
\newcommand{\pto}{\rightharpoonup}
\newcommand{\Un}{\mathcal{U}}

\newcommand{\join}{\vee}

\newcommand{\conv}{^{\mspace{-2mu}\Downarrow\mspace{-2mu}}}

\newcommand{\meaningofconv}[1]{\left\llbracket{#1}\right\rrbracket\conv}

\newcommand{\arrowlift}{\ensuremath{lift}}
\newcommand{\arrowarr}{\ensuremath{arr}}
\newcommand{\arrowcomp}{\ensuremath{{>}\mspace{-6mu}{>}\mspace{-6mu}{>}}}
\newcommand{\arrowpair}{\ensuremath{\mathit{\&\mspace{-7.5mu}\&\mspace{-7.5mu}\&}}}
\newcommand{\arrowif}{\ensuremath{ifte}}
\newcommand{\arrowconvif}{\ensuremath{ifte\conv}}
\newcommand{\arrowlazy}{\ensuremath{lazy}}
\newcommand{\arrowapp}{\ensuremath{app}}
\newcommand{\arrowrun}{\ensuremath{run}}
\newcommand{\arrowget}{\ensuremath{get}}
\newcommand{\arrowerror}{\ensuremath{error}}
\newcommand{\arrowtrans}{\ensuremath{\eta}}

\newcommand{\gen}{_\mathrm{a}}
\newcommand{\genb}{_\mathrm{b}}
\newcommand{\genc}{_\mathrm{a^{\mspace{-2mu}*}}}
\newcommand{\gend}{_\mathrm{b^{\mspace{-2mu}*}}}

\DeclareMathOperator{\botto}{\arrow_{\mspace{-3mu}\bot}}
\newcommand{\arrbot}{\arrowarr_{\mspace{-3mu}\bot}}
\newcommand{\compbot}{\arrowcomp_{\mspace{-5mu}\bot}}
\newcommand{\pairbot}{\arrowpair_{\mspace{-3mu}\bot}}
\newcommand{\ifbot}{\arrowif_{\mspace{-2mu}\bot}}
\newcommand{\lazybot}{\arrowlazy_{\mspace{-2mu}\bot}}

\newcommand{\map}{_\mathrm{map}}
\DeclareMathOperator{\mapto}{\arrow_{\mspace{-21mu}\map}}
\newcommand{\liftmap}{\arrowlift\map}
\newcommand{\arrmap}{\arrowarr\map}
\newcommand{\compmap}{\arrowcomp\map}
\newcommand{\pairmap}{\arrowpair\map}
\newcommand{\ifmap}{\arrowif\map}
\newcommand{\lazymap}{\arrowlazy\map}

\newcommand{\pre}{_\mathrm{pre}}
\DeclareMathOperator{\preto}{\arrow_{\mspace{-19mu}\pre}}
\newcommand{\liftpre}{\arrowlift\pre}
\newcommand{\arrpre}{\arrowarr\pre}
\newcommand{\comppre}{\arrowcomp\pre}
\newcommand{\pairpre}{\arrowpair\pre}
\newcommand{\ifpre}{\arrowif\pre}
\newcommand{\lazypre}{\arrowlazy\pre}

\newcommand{\pbot}{{\bot^{\mspace{-4mu}*}}}
\DeclareMathOperator{\pbotto}{\arrow_{\mspace{-3mu}\pbot}}
\newcommand{\arrpbot}{\arrowarr_{\mspace{-3mu}\pbot}}
\newcommand{\comppbot}{\arrowcomp_{\mspace{-5mu}\pbot}}
\newcommand{\pairpbot}{\arrowpair_{\mspace{-3mu}\pbot}}
\newcommand{\ifpbot}{\arrowif_{\mspace{-2mu}\pbot}}
\newcommand{\convifpbot}{\arrowconvif_{\mspace{-2mu}\pbot}}
\newcommand{\lazypbot}{\arrowlazy_{\mspace{-2mu}\pbot}}

\newcommand{\pmap}{_\mathrm{map^{\mspace{-2mu}*}}}
\DeclareMathOperator{\pmapto}{\arrow_{\mspace{-22mu}_{\mathrm{map*}}}}
\newcommand{\liftpmap}{\arrowlift\pmap}
\newcommand{\arrpmap}{\arrowarr\pmap}
\newcommand{\comppmap}{\arrowcomp\pmap}
\newcommand{\pairpmap}{\arrowpair\pmap}
\newcommand{\ifpmap}{\arrowif\pmap}
\newcommand{\convifpmap}{\arrowconvif\pmap}
\newcommand{\lazypmap}{\arrowlazy\pmap}

\newcommand{\ppre}{_\mathrm{pre^{\mspace{-2mu}*}}}
\DeclareMathOperator{\ppreto}{\arrow_{\mspace{-19mu}_{\mathrm{pre*}}}}
\newcommand{\liftppre}{\arrowlift\ppre}
\newcommand{\arrppre}{\arrowarr\ppre}
\newcommand{\compppre}{\arrowcomp\ppre}
\newcommand{\pairppre}{\arrowpair\ppre}
\newcommand{\ifppre}{\arrowif\ppre}
\newcommand{\convifppre}{\arrowconvif\ppre}
\newcommand{\lazyppre}{\arrowlazy\ppre}

\newcommand{\prepto}{\pto_{\mspace{-19mu}\pre}}

\DeclareMathVersion{sans}
\SetSymbolFont{operators}{sans}{T1}{\sfdefault}{m}{n}
\SetSymbolFont{letters}{sans}{T1}{\sfdefault}{m}{n}
\SetMathAlphabet\mathbf{sans}{T1}{\sfdefault}{bx}{n}
\SetMathAlphabet\mathsf{sans}{T1}{\sfdefault}{m}{n}
\SetMathAlphabet\mathit{sans}{T1}{\rmdefault}{m}{it}

\DeclareSymbolFont{greekletters}{OML}{cmss}{m}{n}
\DeclareMathSymbol{\Gamma}  {\mathord}{greekletters}{"00}
\DeclareMathSymbol{\Delta}  {\mathord}{greekletters}{"01}
\DeclareMathSymbol{\Theta}  {\mathord}{greekletters}{"02}
\DeclareMathSymbol{\Lambda} {\mathord}{greekletters}{"03}
\DeclareMathSymbol{\Xi}     {\mathord}{greekletters}{"04}
\DeclareMathSymbol{\Pi}     {\mathord}{greekletters}{"05}
\DeclareMathSymbol{\Sigma}  {\mathord}{greekletters}{"06}
\DeclareMathSymbol{\Upsilon}{\mathord}{greekletters}{"07}
\DeclareMathSymbol{\Phi}    {\mathord}{greekletters}{"08}
\DeclareMathSymbol{\Psi}    {\mathord}{greekletters}{"09}
\DeclareMathSymbol{\Omega}  {\mathord}{greekletters}{"0A}

\DeclareMathSymbol{\alpha}{\mathord}{greekletters}{"0B}
\DeclareMathSymbol{\beta}{\mathord}{greekletters}{"0C}
\DeclareMathSymbol{\gamma}{\mathord}{greekletters}{"0D}
\DeclareMathSymbol{\delta}{\mathord}{greekletters}{"0E}
\DeclareMathSymbol{\epsilon}{\mathord}{greekletters}{"0F}
\DeclareMathSymbol{\zeta}{\mathord}{greekletters}{"10}
\DeclareMathSymbol{\eta}{\mathord}{greekletters}{"11}
\DeclareMathSymbol{\theta}{\mathord}{greekletters}{"12}
\DeclareMathSymbol{\iota}{\mathord}{greekletters}{"13}
\DeclareMathSymbol{\kappa}{\mathord}{greekletters}{"14}
\DeclareMathSymbol{\lambda}{\mathord}{greekletters}{"15}
\DeclareMathSymbol{\mu}{\mathord}{greekletters}{"16}
\DeclareMathSymbol{\nu}{\mathord}{greekletters}{"17}
\DeclareMathSymbol{\xi}{\mathord}{greekletters}{"18}
\DeclareMathSymbol{\pi}{\mathord}{greekletters}{"19}
\DeclareMathSymbol{\rho}{\mathord}{greekletters}{"1A}
\DeclareMathSymbol{\sigma}{\mathord}{greekletters}{"1B}
\DeclareMathSymbol{\tau}{\mathord}{greekletters}{"1C}
\DeclareMathSymbol{\upsilon}{\mathord}{greekletters}{"1D}
\DeclareMathSymbol{\phi}{\mathord}{greekletters}{"1E}
\DeclareMathSymbol{\chi}{\mathord}{greekletters}{"1F}
\DeclareMathSymbol{\psi}{\mathord}{greekletters}{"20}
\DeclareMathSymbol{\omega}{\mathord}{greekletters}{"21}
\DeclareMathSymbol{\varepsilon}{\mathord}{greekletters}{"22}
\DeclareMathSymbol{\vartheta}{\mathord}{greekletters}{"23}
\DeclareMathSymbol{\varpi}{\mathord}{greekletters}{"24}
\DeclareMathSymbol{\varrho}{\mathord}{greekletters}{"25}
\DeclareMathSymbol{\varsigma}{\mathord}{greekletters}{"26}

\DeclareSymbolFont{fixpunct}{T1}{\rmdefault}{m}{n}
\DeclareSymbolFont{fixpunctoml}{OML}{\rmdefault}{m}{n}
\DeclareMathSymbol{,}{\mathpunct}{fixpunct}{`,}
\DeclareMathSymbol{.}{\mathord}{fixpunct}{`.}
\DeclareMathSymbol{\ldotp}{\mathord}{fixpunct}{`.}
\DeclareMathSymbol{!}{\mathord}{letters}{`-}
\DeclareMathSymbol{/}{\mathpunct}{fixpunct}{`/}
\DeclareMathSymbol{\rightharpoonup}{\mathrel}{fixpunctoml}{"2A}
\DeclareMathSymbol{^}{\mathpunct}{fixpunct}{`!}

\DeclareSymbolFont{numbers}{T1}{pvh}{m}{n}
\SetSymbolFont{numbers}{sans}{T1}{\rmdefault}{m}{n}
\DeclareMathSymbol{0}{\mathalpha}{numbers}{"30}
\DeclareMathSymbol{1}{\mathalpha}{numbers}{"31}
\DeclareMathSymbol{2}{\mathalpha}{numbers}{"32}
\DeclareMathSymbol{3}{\mathalpha}{numbers}{"33}
\DeclareMathSymbol{4}{\mathalpha}{numbers}{"34}
\DeclareMathSymbol{5}{\mathalpha}{numbers}{"35}
\DeclareMathSymbol{6}{\mathalpha}{numbers}{"36}
\DeclareMathSymbol{7}{\mathalpha}{numbers}{"37}
\DeclareMathSymbol{8}{\mathalpha}{numbers}{"38}
\DeclareMathSymbol{9}{\mathalpha}{numbers}{"39}

\renewcommand{\notin}{\not\in}

\newcommand\mathtest[2]{\mathchoice{#1}{#2}{#2}{#2}}

\newenvironment{lzfcenv}[1][l]%
{%
	\begin{array}[t]{@{}#1@{}}%
}%
{%
	\end{array}%
}

\newcommand{\tab}{\ \ \ }

\newcommand{\lzfc}[2][l]{\begin{lzfcenv}[#1]#2\end{lzfcenv}}
\newcommand{\tlzfc}[1]{\ensuremath{#1}}

\newcommand{\lzfcsplit}[2][@{}]{\lzfc[r #1 l]{#2}}

\newcommand{\lzfclet}[2]{\lzfc{let\ \ \lzfc[r@{\ \objdef\ }l]{#1} \\ \lzfc{\tab in\ \ #2}}}
\newcommand{\lzfccond}[2][\ \longrightarrow\ ]{\lzfc{cond\ \ \lzfc[l@{\ {#1}\ }l]{#2}}}
\newcommand{\lzfccase}[3][\ \longrightarrow\ ]{\lzfc{case\ \ {#2}\\\ \ \lzfc[l@{\ {#1}\ }l]{#3}}}

\newcommand{\dash}{\text{-}}

%
% Racket syntax (must be last - or at least \defschememathescape must be - see last line)
%

% DrRacket's default colors
\definecolor{identifiercolor}{rgb}{0.15,0.15,0.5}
\definecolor{keywordcolor}{rgb}{0.0,0.0,0.0}
\definecolor{constantcolor}{rgb}{0.16,0.5,0.15}
\definecolor{parenthesiscolor}{rgb}{0.52,0.24,0.14}

\def\keywordfont#1{\textcolor{keywordcolor}{#1}}
\def\variablefont#1{\textcolor{identifiercolor}{#1}}
\def\constantfont#1{\textcolor{constantcolor}{#1}}
\def\datafont#1{\textcolor{parenthesiscolor}{#1}}

\makeatletter
\def\mycodehook{\tt\color{parenthesiscolor}\@setfontsize\small{10pt}{11pt minus0.5pt}}
\makeatother
\let\schemecodehook\mycodehook
\setlength{\leftcodeskip}{\parindent}

\setlength{\leftcodeskip}{0pt}
\setlength{\rightcodeskip}{0pt plus 1fil}
\setlength{\abovecodeskip}{0.5\abovedisplayskip}
\setlength{\belowcodeskip}{0.5\belowdisplayskip}

\unsetspecialsymbol{.}
\unsetspecialsymbol{...}
\unsetspecialsymbol{-}
\unsetspecialsymbol{1-}
\unsetspecialsymbol{-1+}

\setspecialsymbol{<blank-line>}{\va{}}
\setspecialsymbol{lambda}{\va{$\mathtt{\lambda}$}}
\setspecialsymbol{λ}{\va{$\mathtt{\lambda}$}}
\setspecialsymbol{~}{\va{$\mathtt{\sim}$}}
\setspecialsymbol{_}{\va{$\underbar{ }$}}
\setspecialsymbol{Omega}{\va{$\mathtt{\Omega}$}}
\setspecialsymbol{omega}{\va{$\mathtt{\omega}$}}
\setspecialsymbol{Ω}{\va{$\mathtt{\Omega}$}}
\setspecialsymbol{ω}{\va{$\mathtt{\omega}$}}

\setkeyword{define/drbayes struct/drbayes}

\defschememathescape{$}


\newsavebox{\codebox}

\newcommand{\smallmathfont}{\fontsize{7.5}{9}\selectfont}

\newenvironment{displaybreaks}%
{%
	\begingroup%
	\allowdisplaybreaks%
}%
{%
	\endgroup%
	\ignorespacesafterend%
}

%\excludecomment{proof}

\newcommand{\figref}[1]{Fig.~\ref{#1}}
\newcommand{\figsref}[1]{Figs.~\ref{#1}}

\newcommand{\arrow}{\rightsquigarrow}

\newcommand{\restrict}[1]{\lvert_{#1}}
\newcommand{\pto}{\rightharpoonup}
\newcommand{\Univ}{\mathbb{U}}
\newcommand{\Un}{\mathcal{U}}

\newcommand{\join}{\vee}

\newcommand{\conv}{^{\mspace{-2mu}\Downarrow\mspace{-2mu}}}

\newcommand{\meaningofconv}[1]{\left\llbracket{#1}\right\rrbracket\conv}

\newcommand{\arrowlift}{\ensuremath{lift}}
\newcommand{\arrowarr}{\ensuremath{arr}}
\newcommand{\arrowcomp}{\ensuremath{{>}\mspace{-6mu}{>}\mspace{-6mu}{>}}}
\newcommand{\arrowpair}{\ensuremath{\mathit{\&\mspace{-7.5mu}\&\mspace{-7.5mu}\&}}}
\newcommand{\arrowif}{\ensuremath{ifte}}
\newcommand{\arrowconvif}{\ensuremath{ifte\conv}}
\newcommand{\arrowlazy}{\ensuremath{lazy}}
\newcommand{\arrowapp}{\ensuremath{app}}
\newcommand{\arrowrun}{\ensuremath{run}}
\newcommand{\arrowget}{\ensuremath{get}}
\newcommand{\arrowerror}{\ensuremath{error}}
\newcommand{\arrowtrans}{\ensuremath{\eta}}

\newcommand{\gen}{_\mathrm{a}}
\newcommand{\genb}{_\mathrm{b}}
\newcommand{\genc}{_\mathrm{a^{\mspace{-2mu}*}}}
\newcommand{\gend}{_\mathrm{b^{\mspace{-2mu}*}}}

\DeclareMathOperator{\botto}{\arrow_{\mspace{-3mu}\bot}}
\newcommand{\arrbot}{\arrowarr_{\mspace{-3mu}\bot}}
\newcommand{\compbot}{\arrowcomp_{\mspace{-5mu}\bot}}
\newcommand{\pairbot}{\arrowpair_{\mspace{-3mu}\bot}}
\newcommand{\ifbot}{\arrowif_{\mspace{-2mu}\bot}}
\newcommand{\lazybot}{\arrowlazy_{\mspace{-2mu}\bot}}

\newcommand{\map}{_\mathrm{map}}
\DeclareMathOperator{\mapto}{\arrow_{\mspace{-21mu}\map}}
\newcommand{\liftmap}{\arrowlift\map}
\newcommand{\arrmap}{\arrowarr\map}
\newcommand{\compmap}{\arrowcomp\map}
\newcommand{\pairmap}{\arrowpair\map}
\newcommand{\ifmap}{\arrowif\map}
\newcommand{\lazymap}{\arrowlazy\map}

\newcommand{\pre}{_\mathrm{pre}}
\DeclareMathOperator{\preto}{\arrow_{\mspace{-19mu}\pre}}
\newcommand{\liftpre}{\arrowlift\pre}
\newcommand{\arrpre}{\arrowarr\pre}
\newcommand{\comppre}{\arrowcomp\pre}
\newcommand{\pairpre}{\arrowpair\pre}
\newcommand{\ifpre}{\arrowif\pre}
\newcommand{\lazypre}{\arrowlazy\pre}

\newcommand{\pbot}{{\bot^{\mspace{-4mu}*}}}
\DeclareMathOperator{\pbotto}{\arrow_{\mspace{-3mu}\pbot}}
\newcommand{\arrpbot}{\arrowarr_{\mspace{-3mu}\pbot}}
\newcommand{\comppbot}{\arrowcomp_{\mspace{-5mu}\pbot}}
\newcommand{\pairpbot}{\arrowpair_{\mspace{-3mu}\pbot}}
\newcommand{\ifpbot}{\arrowif_{\mspace{-2mu}\pbot}}
\newcommand{\convifpbot}{\arrowconvif_{\mspace{-2mu}\pbot}}
\newcommand{\lazypbot}{\arrowlazy_{\mspace{-2mu}\pbot}}

\newcommand{\pmap}{_\mathrm{map^{\mspace{-2mu}*}}}
\DeclareMathOperator{\pmapto}{\arrow_{\mspace{-22mu}_{\mathrm{map*}}}}
\newcommand{\liftpmap}{\arrowlift\pmap}
\newcommand{\arrpmap}{\arrowarr\pmap}
\newcommand{\comppmap}{\arrowcomp\pmap}
\newcommand{\pairpmap}{\arrowpair\pmap}
\newcommand{\ifpmap}{\arrowif\pmap}
\newcommand{\convifpmap}{\arrowconvif\pmap}
\newcommand{\lazypmap}{\arrowlazy\pmap}

\newcommand{\ppre}{_\mathrm{pre^{\mspace{-2mu}*}}}
\DeclareMathOperator{\ppreto}{\arrow_{\mspace{-19mu}_{\mathrm{pre*}}}}
\newcommand{\liftppre}{\arrowlift\ppre}
\newcommand{\arrppre}{\arrowarr\ppre}
\newcommand{\compppre}{\arrowcomp\ppre}
\newcommand{\pairppre}{\arrowpair\ppre}
\newcommand{\ifppre}{\arrowif\ppre}
\newcommand{\convifppre}{\arrowconvif\ppre}
\newcommand{\lazyppre}{\arrowlazy\ppre}

\newcommand{\prepto}{\pto_{\mspace{-19mu}\pre}}

\DeclarePairedDelimiter{\ivl}{[\mspace{-4.5mu}(}{)\mspace{-4.5mu}]}


\begin{document}

\special{papersize=8.5in,11in}
\setlength{\pdfpageheight}{\paperheight}
\setlength{\pdfpagewidth}{\paperwidth}

\conferenceinfo{PLDI'14}{June 9--11, 2014, Edinburgh, Scotland, UK}
\copyrightyear{2013}
\copyrightdata{978-1-nnnn-nnnn-n/yy/mm} 
\doi{nnnnnnn.nnnnnnn}

% Uncomment one of the following two, if you are not going for the 
% traditional copyright transfer agreement.

\exclusivelicense                % ACM gets exclusive license to publish, 
                                 % you retain copyright

%\permissiontopublish             % ACM gets nonexclusive license to publish
                                  % (paid open-access papers, 
                                  % short abstracts)

%\titlebanner{banner above paper title}        % These are ignored unless
%\preprintfooter{short description of paper}   % 'preprint' option specified.

\title{Applications of Preimage Analysis}
\subtitle{}

\makeatletter

\if \@preprint
  \authorinfo{\ }{\ }{\ }
\else
  \authorinfo{Neil Toronto\and Jay McCarthy}
             {PLT @ Brigham Young University, Provo, Utah, USA}
             {neil.toronto@gmail.com \and jay@cs.byu.edu}
\fi

\makeatother

\maketitle

\begin{abstract}
XXX: todo
\end{abstract}

\category{CR-number}{subcategory}{third-level}

% general terms are not compulsory anymore, 
% you may leave them out
%\terms
%term1, term2

\keywords
keyword1, keyword2

%%%%%%%%%%%%%%%%%%%%%%%%%%%%%%%%%%%%%%%%%%%%%%%%%%%%%%%%%%%%%%%%%%%%%%%%%%%%%%%%%%%%%%%%%%%%%%%%%%%%%
%%%%%%%%%%%%%%%%%%%%%%%%%%%%%%%%%%%%%%%%%%%%%%%%%%%%%%%%%%%%%%%%%%%%%%%%%%%%%%%%%%%%%%%%%%%%%%%%%%%%%
%%%%%%%%%%%%%%%%%%%%%%%%%%%%%%%%%%%%%%%%%%%%%%%%%%%%%%%%%%%%%%%%%%%%%%%%%%%%%%%%%%%%%%%%%%%%%%%%%%%%%

\mathversion{sans}

\section{Introduction}

(Citation to make Kile happy with BibTeX: \cite{cit:sergey-2013pldi-monadic-abstract})

\section{Finite Program Semantics}

\section{Set Representation}

%%%%%%%%%%%%%%%%%%%%%%%%%%%%%%%%%%%%%%%%%%%%%%%%%%%%%%%%%%%%%%%%%%%%%%%%%%%%%
%%%%%%%%%%%%%%%%%%%%%%%%%%%%%%%%%%%%%%%%%%%%%%%%%%%%%%%%%%%%%%%%%%%%%%%%%%%%%
%%%%%%%%%%%%%%%%%%%%%%%%%%%%%%%%%%%%%%%%%%%%%%%%%%%%%%%%%%%%%%%%%%%%%%%%%%%%%

\section{Preimages Under Primitives}

\newcommand{\cl}[1]{\overline{\vphantom{i}{#1}}}
\newcommand{\sub}[1]{_{_{#1}}}

XXX: lifts are generally uncomputable, but many specific lifts are computable or approximately computable

XXX: something about computing functions backward by computing inverses forward

XXX: obvious how to do this with invertible functions (though infinite endpoints are a little tricky); not obvious how to do the same with two-argument functions

\begin{comment}
Math facts:

Every totally ordered space is Tychonoff: completely regular and Hausdorff (aka T2)
T2 spaces are T1

T1 spaces:
Limit point compactness is equivalent to countable compactness
Singletons are closed sets

T2 spaces:
Every sequence has a unique limit
Compact sets are closed
Every pair of compact sets can be separated by neighborhoods
Continuous functions into T2 spaces are determined by their values on dense subsets

Tychonoff spaces:
Subpaces of Tychonoff spaces are Tychonoff
Products of Tychonoff spaces are Tychonoff

There are no M-point compactifications for R^N, N > 1, M > 1



\end{comment}

\subsection{Invertible Primitives}

We consider only strictly monotone functions on $\Re$.
Further on, we recover more generality by using language conditionals to implement \emph{piecewise} monotone functions.

One reason we consider only strictly monotone functions is that they are easy to make invertible.
Recall that a function is invertible if and only if it is injective (one-to-one) and surjective (onto).

\begin{lemma}
\label{lem:monotone-implies-invertible}
If $f : A \to B$ is strictly montone, $f$ is injective.
If $f$ is additionally surjective, $f$ and its inverse are continuous.
\end{lemma}

Preimages under invertible functions can be computed using their inverses.
We are primarily interested in computing preimages under restricted functions.

\begin{lemma}
\label{lem:invertible-function-preimages}
Let $A' \subseteq A$, $B' \subseteq B$, and $f : A \to B$ have inverse $f^{-1} : B \to A$.
Let $f' := restrict~f~A'$. Then
\begin{equation}
	preimage~f'~B'\ =\ A' \i image~f^{-1}~B'
\end{equation}
\end{lemma}

These facts suggest that we can compute images (or preimages) of intervals under any strictly monotone, surjective $f$ by applying $f$ (or its inverse) to interval endpoints to yield an interval.
This is evident for endpoints in $A$.
Limit endpoints like $+\infty$ require a larger $\cl{f}$ defined on a compact superset of $A$.

The next theorem is easier to state with interval notation in which the kind of interval is not baked into the syntax.

\begin{definition}[interval]
$\ivl{a_1,a_2,\alpha_1,\alpha_2}$ denotes an interval, where $a_1,a_2 \in \cl{\Re}$ are extended real endpoints, and $\alpha_1,\alpha_2 \in Bool$ determine whether $a_1$ and $a_2$ are contained in the interval.
\end{definition}

\begin{example}
Examples of intervals using $\ivl{\cdots}$ notation are
\begin{equation}
\begin{aligned}
	\ivl{0,1,true,false} &= [0,1) \\
	\ivl{-\infty,0,false,true} &= (-\infty,0] \\
	\ivl{-\infty,+\infty,false,false} &= (-\infty,+\infty) = \Re \\
	\ivl{-\infty,+\infty,true,true} &= [-\infty,+\infty] = \cl{\Re}
\end{aligned}
\end{equation}
All but the last are subsets of $\Re$.
\exampleqed
\end{example}

\begin{theorem}[images of intervals by endpoints]
\label{thm:images-of-intervals}
Let $\cl{A}$ and $\cl{B}$ be compact subsets of $\cl{\Re}$, $\cl{f} : \cl{A} \to \cl{B}$ strictly monotone and surjective, and $f$ the restriction of $\cl{f}$ to some $A \subseteq \cl{A}$.
For all nonempty $\ivl{a_1,a_2,\alpha_1,\alpha_2} \subseteq A$,
\begin{itemize}
	\item If $\cl{f}$ is increasing, $image~f~\ivl{a_1,a_2,\alpha_1,\alpha_2} = \ivl{\cl{f}~a_1, \cl{f}~a_2,\alpha_1,\alpha_2}$.
	\item If $\cl{f}$ is decreasing, $image~f~\ivl{a_1,a_2,\alpha_1,\alpha_2} = \ivl{\cl{f}~a_2, \cl{f}~a_1,\alpha_2,\alpha_1}$.
\end{itemize}
\end{theorem}
\begin{proof}
Because $\cl{A}$ is compact and totally ordered, every subset of $\cl{A}$ has a lower and an upper bound in $\cl{A}$.
Therefore, the endpoints of every interval subset of $A$ are in $\cl{A}$.

Let $(a_1,a_2] \subseteq A$.
Suppose $\cl{f}$ is strictly increasing; thus $a_1 < a \leq a_2$ if and only if $\cl{f}~a_1 < \cl{f}~a \leq \cl{f}~a_2$, so $image~f~(a_1,a_2] = image~\cl{f}~(a_1,a_2] = (\cl{f}~a_1,\cl{f}~a_2]$.
The remaining cases are similar.
\end{proof}

To use Theorem~\ref{thm:images-of-intervals} to compute preimages under $f$ by computing images under its inverse $f^{-1}$, we must know whether $f^{-1}$ is increasing or decreasing.
The following lemma can help.

\begin{lemma}
\label{lem:inverse-direction}
If $f : A \to B$ is strictly monotone and surjective with inverse $f^{-1} : B \to A$, then $f$ is increasing if and only if $f^{-1}$ is increasing.
\end{lemma}

\begin{example}
To extend $log : (0,+\infty) \to \Re$ to $[0,+\infty]$, define
\begin{equation}
\begin{aligned}
	\cl{log}~a\ :\!\!&= \lim_{a' \to a} log~a'
\\
	&=\
		\lzfccond{
			a = 0 & -\infty \\
			a = +\infty & +\infty \\
			else & log~a \\
		}
\end{aligned}
\end{equation}
The extension of its inverse $exp$ is $\cl{exp} : \cl{\Re} \to [0,+\infty]$, defined similarly, which by Lemma~\ref{lem:inverse-direction} is also strictly increasing.
Thus,
\begin{equation}
\begin{aligned}
	image~log~(0,1]\ &=\ (\cl{log}~0,\cl{log}~1] = (-\infty,0]
\\[6pt]
	preimage~log~[0,+\infty)
		\ &=\ image~exp~[0,+\infty)
\\
		\ &=\ [\mspace{2mu}\cl{exp}~0,\cl{exp}~{+\infty})
		= [1,+\infty)
\end{aligned}
\end{equation}
by Theorem~\ref{thm:images-of-intervals} and Lemma~\ref{lem:invertible-function-preimages},
\exampleqed
\end{example}

%%%%%%%%%%%%%%%%%%%%%%%%%%%%%%%%%%%%%%%%%%%%%%%%%%%%%%%%%%%%%%%%%%%%%%%%%%%%%
%%%%%%%%%%%%%%%%%%%%%%%%%%%%%%%%%%%%%%%%%%%%%%%%%%%%%%%%%%%%%%%%%%%%%%%%%%%%%
%%%%%%%%%%%%%%%%%%%%%%%%%%%%%%%%%%%%%%%%%%%%%%%%%%%%%%%%%%%%%%%%%%%%%%%%%%%%%

\subsection{Two-Argument Primitives}

We do not expect to be able to compute preimages under $\Re \times \Re \pto \Re$ primitives by simply inverting them.
Two-argument invertible real functions are difficult to define and are usually pathological.

Instead, we compute approximate preimages only, using inverses with respect to one argument (with the other held constant).

\begin{definition}[axial inverse]
\label{def:axial-inverse}
Let $f_c : A \times B \to C$.
Functions $f_a : B \times C \to A$ and $f_b : C \times A \to B$ defined so that
\begin{equation}
	f_c~\pair{a,b} = c\ \iff\ f_a~\pair{b,c} = a\ \iff\ f_b~\pair{c,a} = b
\end{equation}
are \mykeyword{axial inverses} with respect to $f_c$'s first and second arguments.
\end{definition}

We call $f_c$ \mykeyword{axis-invertible} when it has axial inverses $f_a$ and $f_b$.
We call $f_a$ the \mykeyword{first axial inverse} of $f_c$ because it is the inverse of $f_c$ along the first axis: $f_a$ with only $c$ varying (i.e. $\fun{c \in C} f_a~\pair{b,c}$), is the inverse of $f_c$ with only $a$ varying (i.e. $\fun{a \in A} f_c~\pair{a,b}$).
Similarly, $f_b$ is the \mykeyword{second axial inverse}.

TODO: plot of $f_c$ with $b$ fixed?

\begin{example}
\label{ex:plus-axial-inverses}
Let $add_c : \Re \times \Re \to \Re$, $add_c~\pair{a,b} := a+b$.
Its axial inverses are $add_a~\pair{b,c} := c - b$ and $add_b~\pair{c,a} := c - a$.
\exampleqed
\end{example}

We have chosen the axial inverse function types carefully: they are the only types for which $f_c$, $f_a$ and $f_b$ form a cyclic group.

\begin{lemma}
\label{lem:axial-inverse-cyclic-group}
The following statements are equivalent.
\begin{itemize}
	\item $f_c$ has axial inverses $f_a$ and $f_b$.
	\item $f_a$ has axial inverses $f_b$ and $f_c$.
	\item $f_b$ has axial inverses $f_c$ and $f_a$.
\end{itemize}
Equivalently, every axis-invertible function generates a cyclic group of order 3 by inversion in the first axis.
\end{lemma}

This fact is analogous to how mutual inverses $f$ and $f^{-1}$ also form a cyclic group (of order 2, generated by inversion).
Similar to using mutual inversion to compute preimages under both $log$ and $exp$, Lemma~\ref{lem:axial-inverse-cyclic-group} allows computing preimages under two-argument functions related by axial inversion.

\begin{example}
Define $sub_c : \Re \times \Re \to \Re$ by $sub_c~\pair{a,b} := a-b$.
Because $sub_c = add_b$, $sub_a = add_c$ and $sub_b = add_a$.
\exampleqed
\end{example}

Unlike inverses, axial inverses do not provide a direct way to compute exact preimages.
Instead, they provide a way to compute a preimage's smallest rectangular bounding set.

\begin{theorem}[preimage bounds from axial inverse images]
\label{thm:axis-invertible-function-preimages}
Let $A' \subseteq A$, $B' \subseteq B$, $C' \subseteq C$, and $f_c : A \times B \to C$ with axial inverses $f_a$ and $f_b$.
If $f_c' = restrict~f_c~(A' \times B')$, then
\begin{equation}
	preimage~f_c'~C'\ \subseteq\ \lzfcsplit{
		&(A' \i image~f_a~(B' \times C'))\ \times \\
		&(B' \i image~f_b~(C' \times A'))}
\end{equation}
Further, the right-hand side is the smallest rectangular superset.
\end{theorem}
\begin{proof}
The smallest rectangle containing $preimage~f_c'~C'$ is
\begin{equation}
	preimage~f_c'~C'\ \subseteq\ 
		\lzfcsplit{
			&(image~fst~(preimage~f_c'~C'))\ \times \\
			&(image~snd~(preimage~f_c'~C'))}
\end{equation}
Starting with the first set in the product, expand definitions, distribute $fst$, replace $f_c~\pair{a,b} = c$ by $f_a~\pair{b,c} = a$, and simplify:
\begin{displaybreaks}
\begin{align*}
	&image~fst~(preimage~f_c'~C')
\nobreak\\
	&\tab =\ image~fst~\setb{\pair{a,b} \in A' \times B'}{f_c~\pair{a,b} \in C'}
\\
	&\tab =\ \setb{a \in A'}{\Exists{b \in B'} f_c~\pair{a,b} \in C'}
\\
	&\tab =\ \setb{a \in A'}{\Exists{b \in B',c \in C'} f_c~\pair{a,b} = c}
\\
	&\tab =\ \setb{a \in A'}{\Exists{b \in B',c \in C'} f_a~\pair{b,c} = a}
\\
	&\tab =\ \setb{f_a~\pair{b,c}}{b \in B', c \in C', f_a~\pair{b,c} \in A'}
\\
	&\tab =\ A' \i \setb{f_a~\pair{b,c}}{b \in B', c \in C'}
\nobreak\\
	&\tab =\ A' \i image~f_a~(B' \times C')
\end{align*}
\end{displaybreaks}
The second set in the product is similar.
\end{proof}

\begin{example}
Let $add_c' := restrict~add_c~([0,1] \times [0,2])$.
By Theorem~\ref{thm:axis-invertible-function-preimages},
\begin{align*}
	preimage~add_c'~[0,\tfrac{1}{2}]
		&\subseteq \lzfcsplit{
			&([0,1] \i image~add_a~([0,2] \times [0,\tfrac{1}{2}]))\ \times \\
			&([0,2] \i image~add_b~([0,\tfrac{1}{2}] \times [0,1]))}
\\
		&= ([0,1] \i [-2,\tfrac{1}{2}]) \times ([0,2] \i [-1,\tfrac{1}{2}])
\\
		&= [0,\tfrac{1}{2}] \times [0,\tfrac{1}{2}]
\end{align*}
is the smallest rectangular subset of $[0,1] \times [0,2]$ that contains the preimage of $[0,\tfrac{1}{2}]$ under $add_c$.
\exampleqed
\end{example}

At this point, we have an analogue of Lemma~\ref{lem:invertible-function-preimages}, in that we can compute (approximate) preimages by computing images under (axial) inverses.
To compute images using interval endpoints, we need analogues of Lemma~\ref{lem:monotone-implies-invertible} (strictly monotone, surjective functions are invertible and continuous), Theorem~\ref{thm:images-of-intervals} (images of intervals by endpoints), and Lemma~\ref{lem:inverse-direction} (inverse direction).

We first need a notion of properties that hold along an axis for every fixed value of the other argument.

\begin{definition}[uniform axis property]
$f_c : A \times B \to C$ has property $P$ \keyword{uniformly} in its first axis when $P~(flip~(curry~f_c)~b)$ for all $b \in B$, and uniformly in its second axis when $P~(curry~f_c~a)$ for all $a \in A$.
If the axis is not specified, $P$ holds uniformly for both.
\end{definition}

Now Lemma~\ref{lem:monotone-implies-invertible}'s analogue is an easy corollary.

\begin{lemma}
\label{lem:uniformly-monotone-implies-invertible}
Let $f_c : A \times B \to C$ for totally ordered $A$, $B$ and $C$.
If $f_c$ is uniformly surjective and either uniformly strictly increasing or uniformly strictly decreasing in each axis, then $f_c$ is axis-invertible; further, it and its axial inverses are continuous.
\end{lemma}

From here on, assume axis monotonicity properties are uniform unless otherwise stated.

\begin{example}
$add_c$ is uniformly surjective and strictly increasing.
$sub_c$ is uniformly surjective and strictly increasing/decreasing in its first/second axis.
Therefore, both are axis-invertible.
\exampleqed
\end{example}

Restriction usually makes a function not uniformly surjective.

\begin{example}
Let $add_c' : [0,1] \times [0,1] \to [0,2]$, defined by restricting $add_c$.
It is strictly increasing, but not uniformly surjective: the range of $curry~add_c'~0$ is $[0,1]$, not $[0,2]$.
\exampleqed
\end{example}

Fortunately, restriction sometimes does the opposite.

\begin{example}
Define $mul_c : \Re \times \Re \to \Re$ by $mul_c~\pair{a,b} := a \cdot b$.
It is not uniformly surjective nor strictly monotone because $mul_c~\pair{0,b} = 0$ for all $b \in B$.
But $mul_c^{++} : (0,+\infty) \times (0,+\infty) \to (0,+\infty)$, and $mul_c$ restricted to the other quadrants, are uniformly surjective and strictly increasing or decreasing in each axis.
\exampleqed
\end{example}

Theorem~\ref{thm:images-of-intervals} justifies computing images of intervals with infinite endpoints by applying an extended function to the endpoints.
Its two-argument analogue is more involved because extended, two-argument functions may not be defined at every point.

\begin{example}
$add_c$ cannot be extended to $\cl{add_c} : \cl{\Re} \times \cl{\Re} \to \cl{\Re}$ in the same way $log$ is extended to $\cl{log}$ because
\begin{equation}
	\lim_{\pair{a',b'} \to \pair{a,b}} add_c~\pair{a',b'}
\end{equation}
does not exist when $\pair{a,b}$ is $\pair{-\infty,+\infty}$ or $\pair{+\infty,-\infty}$.
\exampleqed
\end{example}

The previous example suggests that extensions of increasing, two-argument functions are always well-defined except at off-diagonal corners.
This is true, and similar statements hold for axes with other directions, and for more restricted domains.

\begin{theorem}
\label{thm:two-argument-extensions}
Let $A,B,C \subseteq \Re$ and $f_c : A \times B \to C$ be uniformly surjective and strictly increasing or decreasing in each axis.
Let $\cl{A}$, $\cl{B}$ and $\cl{C}$ be the closures of $A$, $B$ and $C$ in $\cl{\Re}$.
The following extension is well-defined:
\begin{equation}
\begin{aligned}
	&\cl{f_c} : (\cl{A} \times \cl{B}) \w N \to \cl{C} \\
	&\cl{f_c}~\pair{a,b}\ :=\ \lim_{\pair{a',b'} \to \pair{a,b}} f_c~\pair{a',b'}
\end{aligned}
\end{equation}
where $N := \set{\pair{min~\cl{A},max~\cl{B}},\pair{max~\cl{A},min~\cl{B}}}$ if $f_c$ is increasing or decreasing, and
$N := \set{\pair{min~\cl{A},min~\cl{B}},\pair{max~\cl{A},max~\cl{B}}}$ if $f_c$ is increasing/decreasing or decreasing/increasing.
\end{theorem}
\begin{proof}
Suppose $f_c$ is increasing, and let $g : \Nat \to A \times B$ be a sequence of $f_c$'s domain values.

Any $g$ that converges to $\pair{max~\cl{A},max~\cl{B}}$ has a strictly increasing subsequence.
By monotonicity, $map~f_c~g$ has a strictly increasing subsequence.
It is bounded by $max~\cl{C}$, so $\cl{f_c}~\pair{max~\cl{A},max~\cl{B}} = max~\cl{C}$.
A similar argument proves $\cl{f}~\pair{min~\cl{A},min~\cl{B}} = min~\cl{C}$.

For any $g$ that converges to $\pair{max~\cl{A},b'}$ for some $b' \in B$, define
\begin{equation}
	g'\ :=\ map~(\fun{\pair{a,b}}\pair{f_a~\pair{b',f_c~\pair{a,b}}',b'})~g
\end{equation}
where $f_a$ is $f_c$'s first axial inverse, so that $map~f_c~g = map~f_c~g'$.
Because $g'$ has a subsequence that is strictly increasing in the first of each pair, and because the second of each pair is the constant $b'$, by monotonicity, $map~f_c~g'$ has a strictly increasing subsequence.
It is bounded by $max~\cl{C}$, so $\cl{f_c}~\pair{max~\cl{A},b'} = max~\cl{C}$.
By similar arguments, $\cl{f}~\pair{min~\cl{A},b} = min~\cl{C}$ and so on.

Arguments for $f$ decreasing, etc., are similar.
\end{proof}

Following the proof of Theorem~\ref{thm:two-argument-extensions}, extensions of two-argument functions can be defined by two corner cases, four border cases, and an interior case.

\begin{example}
Define $pow_c : (0,1) \times (0,+\infty) \to (0,1)$ by $pow_c~\pair{a,b} := exp~(b \cdot log~a)$, which is increasing/decreasing.
Its extension to a subset of $\cl{\Re} \times \cl{\Re}$ is
\begin{equation}
\begin{aligned}
	&\cl{pow_c} : ([0,1] \times [0,+\infty]) \w N \to [0,1] \\
	&\cl{pow_c}~\pair{a,b}\ :=\
		\lzfccase{\pair{a,b}}{
			\pair{0,+\infty} & 0 \\
			\pair{1,0} & 1 \\ 
			\pair{0,b} & 0 \\
			\pair{1,b} & 1 \\
			\pair{a,0} & 1 \\
			\pair{a,+\infty} & 0 \\
			else & pow_c~\pair{a,b}
		}
\end{aligned}
\end{equation}
where $N := \set{\pair{0,0},\pair{1,+\infty}}$.
\exampleqed
\end{example}

The analogue of Theorem~\ref{thm:images-of-intervals} is easiest to state if we have predicates that indicate a function's direction in each axis.
Define $inc_1 : (A \times B \to C) \tto Bool$ so that $inc_1~f$ if and only if $f$ is strictly increasing in its first axis, and similarly $inc_2$ so that $inc_2~f$ if and only if $f$ is strictly increasing in its second axis.

\begin{theorem}[images of rectangles by interval endpoints]
\label{thm:images-of-rectangles}
Let $A,B,C$ be open subsets of $\Re$, and $f_c : A \times B \to C$ be uniformly surjective and strictly increasing or decreasing in each axis, with extension $\cl{f_c}$ as defined in Theorem~\ref{thm:two-argument-extensions}.
If $A' := \ivl{a_1,a_2,\alpha_1,\alpha_2} \subseteq A$ and $B' := \ivl{b_1,b_2,\beta_1,\beta_2} \subseteq B$, then the image $C'$ under $f_c$ is
\begin{equation}
\begin{aligned}
	&image~f_c~(\ivl{a_1,a_2,\alpha_1,\alpha_2} \times \ivl{b_1,b_2,\beta_1,\beta_2})
\\
	&\tab =\ \lzfclet{
		\pair{a_1,a_2,\alpha_1,\alpha_2} & \lzfccond{(inc_1~f_c) & \pair{a_1,a_2,\alpha_1,\alpha_2} \\ else & \pair{a_2,a_1,\alpha_2,\alpha_1}} \\
		\pair{b_1,b_2,\beta_1,\beta_2} & \lzfccond{(inc_2~f_c) & \pair{b_1,b_2,\beta_1,\beta_2} \\ else & \pair{b_2,b_1,\beta_2,\beta_1}}
	}{\ivl{\cl{f_c}~\pair{a_1,b_1},\cl{f_c}~\pair{a_2,b_2},\alpha_1~and~\beta_1,\alpha_2~and~\beta_2}}
\end{aligned}
\end{equation}
\end{theorem}
\begin{proof}
Because $f_c$ is continuous and $A' \times B'$ is a connected set, $C'$ is a connected set, which in $\Re$ is an interval.
Thus, we need to determine only its bounds, and whether it contains each endpoint.

Suppose $f_c$ is increasing.
By monotonicity, $C'$ is bounded by $\cl{f_c}~\pair{a_1,b_1}$ on the bottom and $\cl{f_c}~\pair{a_2,b_2}]$ on the top.
It is therefore contained in $[\cl{f_c}~\pair{a_1,b_1},\cl{f_c}~\pair{a_2,b_2}]$.
If $\alpha_1$ is $false$, $C'$ cannot contain $\cl{f_c}~\pair{a_1,b_1}$, and if $\beta_1$ is $false$, it cannot contain $\cl{f_c}~\pair{a_1,b_1}$. 
Similarly, if either $\alpha_2$ or $\beta_2$ is $false$, it cannot contain $\cl{f_c}~\pair{a_2,b_2}$.
Thus, $C' = \ivl{\cl{f_c}~\pair{a_1,b_1},\cl{f_c}~\pair{a_2,b_2},\alpha_1~and~\beta_1,\alpha_2~and~\beta_2}$.

We still must prove $\pair{a_1,b_1}$ and $\pair{a_2,b_2}$ are in $\cl{f_c}$'s domain.
First, recall $\cl{f_c} : (\cl{A} \times \cl{B}) \w N \to \cl{C}$, where $\cl{A}$, $\cl{B}$ and $\cl{C}$ are the closures of $A$, $B$ and $C$ in $\cl{\Re}$, and $N = \set{\pair{min~\cl{A},max~\cl{B}},\pair{max~\cl{A},min~\cl{B}}}$.

Because $A' \subseteq A$ and $B' \subseteq A$, $a_1 \neq max~\cl{A}$, $a_2 \neq \min~\cl{A}$, $b_1 \neq max~\cl{B}$, and $b_2 \neq min~\cl{B}$, so
\begin{equation}
\begin{aligned}
	\pair{a_1,b_1} &\neq \pair{max~\cl{A},b} & \text{for all}\ b \in \cl{B} \\
	\pair{a_1,b_1} &\neq \pair{a,max~\cl{B}} & \text{for all}\ a \in \cl{A} \\
	\pair{a_2,b_2} &\neq \pair{min~\cl{A},b} & \text{for all}\ b \in \cl{B} \\
	\pair{a_2,b_2} &\neq \pair{a,min~\cl{B}} & \text{for all}\ a \in \cl{A} \\
\end{aligned}
\end{equation}
Therefore, $\pair{a_1,b_1} \not\in N$ and $\pair{a_2,b_2} \not\in N$, as desired.

The remaining cases for $f_c$ are similar.
\end{proof}

\begin{example}
Because $inc_1~pow_c$ and $not~(inc_2~pow_c)$,
\begin{align*}
	&image~pow_c~((0,\tfrac{1}{2}] \times [2,+\infty))
\\
	&\ =\ \lzfclet{
		\pair{a_1,a_2,\alpha_1,\alpha_2} & \pair{0,\tfrac{1}{2},false,true} \\
		\pair{b_1,b_2,\beta_1,\beta_2} & \pair{+\infty,2,false,true}
	}{\ivl{\cl{pow_c}~\pair{a_1,b_1},\cl{pow_c}~\pair{a_2,b_2},\alpha_1~and~\beta_1,\alpha_2~and~\beta_2}}
\\
	&\ =\ \ivl{\cl{pow_c}~\pair{0,+\infty},\cl{pow_c}~\pair{\tfrac{1}{2},2},false~and~false,true~and~true}
\\
	&\ =\ \ivl{0,\tfrac{1}{4},false,true}\ =\ (0,\tfrac{1}{4}]
\end{align*}
\exampleqed
\end{example}

To use Theorem~\ref{thm:images-of-rectangles} to compute approximate preimages under $f_c$ by computing images under its axial inverses, we must know whether each axis of $f_a$ and $f_b$ is increasing or decreasing.
It helps to have an analogue of Lemma~\ref{lem:inverse-direction} (inverse direction).

\begin{theorem}
Let $f_c : A \times B \to C$ be uniformly surjective and strictly increasing or decreasing in each axis, with axial inverses $f_a$ and $f_b$.
The following statements hold:
\begin{enumerate}
	\item $inc_2~f_a$ if and only if $inc_1~f_c$
	\item $inc_1~f_a$ if and only if $(inc_1~f_c)~xor~(inc_2~f_c)$
\end{enumerate}
\end{theorem}
\begin{proof}
For statement 1, fix $b \in B$ and apply Lemma~\ref{lem:inverse-direction}.

For statement 2, let $c \in C$, $b_1,b_2 \in B$, $a_1 := f_a~\pair{b_1,c}$ and $a_2 := f_a~\pair{b_2,c}$.
Let $c' := f_c~\pair{a_1,b_2}$; note $c = f_c~\pair{a_1,b_1} = f_c~\pair{a_2,b_2}$.
If $inc_1~f_c$ and $inc_2~f_c$, then $a_1 > a_2 \iff c < c'$ and $b_1 < b_2 \iff c < c'$; therefore $b_1 < b_2 \iff a_1 > a_2$.
%If $inc_1~f_c$ and not $inc_2~f_c$, then $a_1 < a_2 \iff c > c'$ and $b_1 < b_2 \iff c > c'$; therefore $b_1 < b_2 \iff a_1 < a_2$.
The remaining cases are similar.
\end{proof}

By Lemma~\ref{lem:axial-inverse-cyclic-group}, $inc_1~f_b$ if and only if $inc_2~f_c$, and $inc_2~f_b$ if and only if $inc_1~f_a$.
We can therefore determine the uniform directions of $f_a$'s and $f_b$'s axes solely from the uniform directions of $f_c$'s axes.

%%%%%%%%%%%%%%%%%%%%%%%%%%%%%%%%%%%%%%%%%%%%%%%%%%%%%%%%%%%%%%%%%%%%%%%%%%%%%
%%%%%%%%%%%%%%%%%%%%%%%%%%%%%%%%%%%%%%%%%%%%%%%%%%%%%%%%%%%%%%%%%%%%%%%%%%%%%
%%%%%%%%%%%%%%%%%%%%%%%%%%%%%%%%%%%%%%%%%%%%%%%%%%%%%%%%%%%%%%%%%%%%%%%%%%%%%

\subsection{Discontinuous Primitives}

\subsection{Primitive Implementation}
\label{sec:primitive-implementation}

Because floating-point functions are defined on subsets of $\cl{\Re}$, it would seem we could compute preimages under strictly monotone, real functions by applying their floating-point counterparts to interval endpoints.
This is mostly true, but we must take care to round in the right directions and to account for floating-point negative zero and not-a-number results.

\begin{equation}
\begin{aligned}
	&\cl{pos!recip} : [0,+\infty] \to [0,+\infty] \\
	&\cl{pos!recip}~a\ =\
		\lzfccond{
			0 < a < +\infty & 1/a \\
			a = 0 & +\infty \\
			a = +\infty & 0
		}
\end{aligned}
\end{equation}

\section{Application: Floating-Point Error Analysis}

\section{Recursive, Probabilistic Program Semantics}

\section{Application: Sampling With Constraints}

\section{Related Work}

\section{Conclusions and Future Work}

\begin{figure*}[!tb]\centering
\smallmathfont
\begin{align*}
	\mathit{p} &\ ::\equiv \ \mathit{x := e};\ ...\ ; \mathit{e}
\\
	\mathit{e} &\ ::\equiv \ \mathit{x~e}\ |\ let~\mathit{e~e}\ |\ env~\mathit{n}\ |\ \mathit{\pair{e,e}}\ |\ fst~\mathit{e}\ |\ snd~\mathit{e}\ |\ if~\mathit{e~e~e}\ |\ \mathit{v}
\\
	\mathit{v} &\ ::\equiv \ \text{[first-order constants]}
\\[-6pt]
\end{align*}
\begin{align*}
\begin{aligned}[t]
	\meaningof{\mathit{x} := \mathit{e};\ ...\ ; \mathit{e_b}}\gen &\ :\equiv\
		\mathit{x} := \meaningof{\mathit{e}}\gen;\ ...\ ; \meaningof{\mathit{e_b}}\gen
\\[3pt]
	\meaningof{\mathit{x}~\mathit{e}}\gen &\ :\equiv\
		\meaningof{\pair{\mathit{e},\pair{}}}\gen~\arrowcomp\gen~\mathit{x}
\\
	\meaningof{\pair{\mathit{e}_1,\mathit{e}_2}}\gen &\ :\equiv\
		\meaningof{\mathit{e}_1}\gen~\arrowpair\gen~\meaningof{\mathit{e}_2}\gen
\\
	\meaningof{fst~\mathit{e}}\gen &\ :\equiv\
		\meaningof{\mathit{e}}\gen~\arrowcomp\gen~\arrowarr\gen~fst
\\
	\meaningof{snd~\mathit{e}}\gen &\ :\equiv\
		\meaningof{\mathit{e}}\gen~\arrowcomp\gen~\arrowarr\gen~snd
\\
	\meaningof{\mathit{v}}\gen &\ :\equiv\ \arrowarr\gen~(const~\mathit{v})
\\[6pt]
	id &\ := \ \fun{a} a
\\
	const~b &\ := \ \fun{a} b
\\
\end{aligned}
&\tab\tab\ 
\begin{aligned}[t]
\\[3pt]
	\meaningof{let~\mathit{e}~\mathit{e_b}}\gen &\ :\equiv\ 
		(\meaningof{\mathit{e}}\gen~\arrowpair\gen~\arrowarr\gen~id)~
			\arrowcomp\gen~
		\meaningof{\mathit{e_b}}\gen
\\
	\meaningof{env~0}\gen &\ :\equiv\ \arrowarr\gen~fst
\\
	\meaningof{env~(\mathit{n}+1)}\gen &\ :\equiv\ \arrowarr\gen~snd~\arrowcomp\gen~\meaningof{env~\mathit{n}}\gen
\\
	\meaningof{if~\mathit{e_c}~\mathit{e_t}~\mathit{e_f}}\gen &\ :\equiv\
		\arrowif\gen~
			\meaningof{\mathit{e_c}}\gen~
			\meaningof{lazy~\mathit{e_t}}\gen~
			\meaningof{lazy~\mathit{e_f}}\gen
\\
	\meaningof{lazy~\mathit{e}}\gen &\ :\equiv\ \arrowlazy\gen~\fun{0}{\meaningof{\mathit{e}}\gen}
\\
\\
	\text{subject to} &\ \meaningof{\mathit{p}}\gen : \pair{} \arrow\gen y \ \text{for some $y$}
\end{aligned}
\end{align*}
\bottomhrule
\caption[ ]{Interpretation of a let-calculus with first-order definitions and De-Bruijn-indexed bindings as arrow $\mathrm{a}$ computations.
}
\label{fig:semantic-function}
\end{figure*}

\begin{figure*}[!tb]\centering
\smallmathfont
\begin{align*}
\begin{aligned}[t]
	&\begin{aligned}[t]
		&X \botto Y \ ::= \ X \tto Y_\bot
	\end{aligned} \\
\\[-6pt]
	&\begin{aligned}[t]
		&\arrbot : (X \tto Y) \tto (X \botto Y) \\
		&\arrbot~f \ := \ f
	\end{aligned} \\
\\[-6pt]
	&\begin{aligned}[t]
		&(\compbot) : (X \botto Y) \tto (Y \botto Z) \tto (X \botto Z) \\
		&(f_1~\compbot~f_2)~a \ := \ if~(f_1~a = \bot)~\bot~(f_2~(f_1~a))
	\end{aligned} \\
\\[-6pt]
	&\begin{aligned}[t]
		&(\pairbot) : (X \botto {Y_1}) \tto (X \botto {Y_2}) \tto (X \botto \pair{Y_1,Y_2}) \\
		&(f_1~\pairbot~f_2)~a \ := \ if~(f_1~a = \bot~or~f_2~a = \bot)~\bot~{\pair{f_1~a,f_2~a}}
	\end{aligned}
\end{aligned}
&\tab\tab
\begin{aligned}[t]
	&\begin{aligned}[t]
		&\ifbot : \lzfcsplit{&(X \botto Bool) \tto (X \botto Y) \tto \\ &(X \botto Y) \tto (X \botto Y)} \\
		&\lzfcsplit{&\ifbot~f_1~f_2~f_3~a \ := \ \\
			&\tab\lzfccase{f_1~a}{true & f_2~a \\ false & f_3~a \\ \bot & \bot}}
	\end{aligned} \\
\\[-6pt]
	&\begin{aligned}[t]
		&\lazybot : (1 \tto (X \botto Y)) \tto (X \botto Y) \\
		&\lazybot~f~a \ := \ f~0~a
	\end{aligned}
\end{aligned}
\end{align*}
\bottomhrule
\caption[ ]{Bottom arrow definitions.}
\label{fig:bottom-arrow-defs}
\end{figure*}

\begin{figure*}[!tb]\centering
\smallmathfont
\begin{align*}
&\begin{aligned}[t]
	&\begin{aligned}[t]
		&X \prepto Y ::= \pair{Set~Y, Set~Y \tto Set~X}
	\end{aligned} \\
\\[-6pt]
	&\begin{aligned}[t]
		&pre : (X \botto Y) \tto (X \prepto Y) \\
		&\lzfcsplit{&pre~f~A \ := \ \\ &\tab\pair{image_\bot~f~A, \fun{B}{preimage_\bot~f~A~B}}}
	\end{aligned} \\
\\[-6pt]
	&\begin{aligned}[t]
		&\emptyset\pre \ := \ \pair{\emptyset,\fun{B} \emptyset}
	\end{aligned} \\
\\[-6pt]
	&\begin{aligned}[t]
		&ap\pre : (X \prepto Y) \tto Set~Y \tto Set~X \\
		&ap\pre~\pair{Y',p}~B \ := \ p~(B \i Y') 
	\end{aligned} \\
\\[-6pt]
	&\begin{aligned}[t]
		&range\pre : (X \prepto Y) \tto Set~Y \\
		&range\pre~\pair{Y',p} \ := \ Y'
	\end{aligned}
\end{aligned}
&&\begin{aligned}[t]
	&\begin{aligned}[t]
		&\pair{\cdot,\cdot}\pre : (X \prepto Y_1) \tto (X \prepto Y_2) \tto (X \prepto Y_1 \times Y_2) \\
		&\pair{\pair{Y_1',p_1},\pair{Y_2',p_2}}\pre \ := \ \\
		&\tab\lzfclet{
			Y' & Y_1' \times Y_2' \\
			p & \fun{B}{\U\limits_{\pair{b_1,b_2} \in B}(p_1~\set{b_1}) \i (p_2~\set{b_2})} \\
		}{\pair{Y',p}}
	\end{aligned} \\
\\[-6pt]
	&\begin{aligned}[t]
		&(\circ\pre) : (Y \prepto Z) \tto (X \prepto Y) \tto (X \prepto Z) \\
		&\pair{Z',p_2} \circ\pre h_1 \ := \ \pair{Z', \fun{C}{ap\pre~h_1~(p_2~C)}}
	\end{aligned} \\
\\[-6pt]
	&\begin{aligned}[t]
		&(\uplus\pre) : (X \prepto Y) \tto (X \prepto Y) \tto (X \prepto Y) \\
		&\lzfcsplit{
			&h_1 \uplus\pre h_2 \ := \ 
			\lzfclet{
					Y' & (range\pre~h_1) \u (range\pre~h_2) \\
					p & \fun{B}{(ap\pre~h_1~B) \u (ap\pre~h_2~B)}
				}{\pair{Y',p}}
		}
	\end{aligned}
\end{aligned}
\\[-6pt]
&\begin{aligned}[t]
\hline
\\[-8pt]
	&\begin{aligned}[t]
		&image_\bot : (X \botto Y) \tto Set~X \tto Set~Y \\
		&image_\bot~f~A \ := \ (image~f~A) \w \set{\bot}
	\end{aligned}
\end{aligned}
&&\begin{aligned}[t]
\hline
\\[-8pt]
	&\begin{aligned}[t]
		&preimage_\bot : (X \botto Y) \tto Set~X \tto Set~Y \tto Set~X \\
		&preimage_\bot~f~A~B \ := \ \setb{a \in A}{f~a \in B}
	\end{aligned}
\end{aligned}
\end{align*}
\bottomhrule
\caption[ ]{Lazy preimage mappings and operations.}
\label{fig:preimage-mapping-defs}
\end{figure*}

\begin{figure*}[!tb]\centering
\smallmathfont
\begin{align*}
\begin{aligned}[t]
	&\begin{aligned}[t]
		&X \preto Y ::= Set~X \tto (X \prepto Y)
	\end{aligned} \\
\\[-6pt]
	&\begin{aligned}[t]
		&\arrpre : (X \tto Y) \tto (X \preto Y) \\
		&\arrpre \ := \ \liftpre \circ \arrbot
	\end{aligned} \\
\\[-6pt]
	&\begin{aligned}[t]
		&(\comppre) : (X \preto Y) \tto (Y \preto Z) \tto (X \preto Z) \\
		&(h_1~\comppre~h_2)~A \ := \ 
			\lzfclet{
				h_1' & h_1~A \\
				h_2' & h_2~(range\pre~h_1')
			}{h_2' \circ\pre h_1'}
	\end{aligned} \\
\\[-6pt]
	&\begin{aligned}[t]
		&(\pairpre) : (X \preto Y) \tto (X \preto Z) \tto (X \preto Y \times Z) \\
		&(h_1~\pairpre~h_2)~A \ := \ \pair{h_1~A,h_2~A}\pre
	\end{aligned}
\end{aligned}
&\tab\tab\tab
\begin{aligned}[t]
	&\begin{aligned}[t]
		&\ifpre : \lzfcsplit{&(X \preto Bool) \tto (X \preto Y) \tto \\ &(X \preto Y) \tto (X \preto Y)} \\
		&\lzfcsplit{&\ifpre~h_1~h_2~h_3~A \ := \ \\
			&\tab\lzfclet{
				h_1' & h_1~A \\
				h_2' & h_2~(ap\pre~h_1'~\set{true}) \\
				h_3' & h_3~(ap\pre~h_1'~\set{false})
			}{h_2' \uplus\pre h_3'}}
	\end{aligned} \\
\\[-6pt]
	&\begin{aligned}[t]
		&\lazypre : (1 \tto (X \preto Y)) \tto (X \preto Y) \\
		&\lazypre~h~A \ := \ if~(A = \emptyset)~\emptyset\pre~(h~0~A)
	\end{aligned} \\
\\[-8pt]
\hline
\\[-4pt]
	&\begin{aligned}[t]
		&\liftpre \ := \ pre
	\end{aligned}
\end{aligned}
\end{align*}
\bottomhrule
\caption[ ]{Preimage arrow definitions.}
\label{fig:preimage-arrow-defs}
\end{figure*}

\begin{figure*}[!tb]\centering
\smallmathfont
\begin{align*}
\begin{aligned}[t]
	&\begin{aligned}[t]
		x \arrow\genc y \ ::= \ AStore~s~(x \arrow\gen y) \ ::= \ J \tto (\pair{s,x} \arrow\gen y)
	\end{aligned} \\
\\[-6pt]
	&\begin{aligned}[t]
		&\arrowarr\genc : (x \tto y) \tto (x \arrow\genc y) \\
		&\arrowarr\genc \ := \ \arrowtrans\genc \circ \arrowarr\gen
	\end{aligned} \\
\\[-6pt]
	&\begin{aligned}[t]
		&(\arrowcomp\genc) : (x \arrow\genc y) \tto (y \arrow\genc z) \tto (x \arrow\genc z) \\
		&(k_1~\arrowcomp\genc~k_2)~j \ := \\
			&\tab(\arrowarr\gen~fst~\arrowpair\gen~k_1~(left~j))~\arrowcomp\gen~k_2~(right~j)
	\end{aligned} \\
\\[-6pt]
	&\begin{aligned}[t]
		&(\arrowpair\genc) : (x \arrow\genc y_1) \tto (x \arrow\genc y_2) \tto (x \arrow\genc \pair{y_1,y_2}) \\
		&(k_1~\arrowpair\genc~k_2)~j \ := \ k_1~(left~j)~\arrowpair\gen~k_2~(right~j)
	\end{aligned} \\
\end{aligned}
&\tab\tab\tab
\begin{aligned}[t]
	&\begin{aligned}[t]
		&\arrowif\genc : \lzfcsplit{&(x \arrow\genc Bool) \tto (x \arrow\genc y) \tto \\ &(x \arrow\genc y) \tto (x \arrow\genc y)} \\
		&\lzfcsplit{&\arrowif\genc~k_1~k_2~k_3~j \ := \ \\
			&\tab\lzfcsplit{\arrowif\gen~&(k_1~(left~j)) \\ &(k_2~(left~(right~j))) \\ &(k_3~(right~(right~j)))}}
	\end{aligned} \\
\\[-6pt]
	&\begin{aligned}[t]
		&\arrowlazy\genc : (1 \tto (x \arrow\genc y)) \tto (x \arrow\genc y) \\
		&\arrowlazy\genc~k~j \ := \ \arrowlazy\gen~\fun{0}{k~0~j}
	\end{aligned} \\
\\[-8pt]
\hline
\\[-8pt]
	&\begin{aligned}[t]
		&\arrowtrans\genc : (x \arrow\gen y) \tto (x \arrow\genc y) \\
		&\arrowtrans\genc~f~j \ := \ \arrowarr\gen~snd~\arrowcomp\gen~f
	\end{aligned}
\end{aligned}
\end{align*}
\bottomhrule
\caption[ ]{$AStore$ (associative store) arrow transformer definitions.}
\label{fig:astore-arrow-defs}
\end{figure*}

\begin{figure*}[!tb]\centering
\smallmathfont
\begin{align*}
\begin{aligned}[t]
	&\begin{aligned}[t]
		id\pre~A &\ := \ \pair{A,\fun{B}{B}} \\
		fst\pre~A &\ := \ \pair{proj_1~A,unproj_1~A} \\
		snd\pre~A &\ := \ \pair{proj_2~A,unproj_2~A} \\
	\end{aligned} \\
\\[-8pt]
\hline
\\[-8pt]
	&\begin{aligned}[t]
		&proj_1 := image~fst \\
		&proj_2 := image~snd \\
		&unproj_1~A~B \ := \ A \i (B \times proj_2~A) \\
		&unproj_2~A~B \ := \ A \i (proj_1~A \times B)
	\end{aligned}
\end{aligned}
&\tab\tab\tab\tab\tab
\begin{aligned}[t]
	&\begin{aligned}[t]
		const\pre~b~A &\ := \ \pair{\set{b},\fun{B}{if~(B = \emptyset)~\emptyset~A}} \\
		\pi\pre~j~A &\ := \ \pair{proj~j~A, unproj~j~A}
	\end{aligned} \\
\\[-8pt]
\hline
\\[-8pt]
	&\begin{aligned}[t]
		&proj : J \tto Set~(J \to X) \tto Set~X \\
		&proj~j~A \ := \ image~(\pi~j)~A
	\end{aligned} \\
\\[-6pt]
	&\begin{aligned}[t]
		&unproj : J \tto Set~(J \to X) \tto Set~X \tto Set~(J \to X) \\
		&unproj~j~A~B \ := \ A \i \prod_{i \in J} if~(j = i)~B~(proj~j~A)
	\end{aligned}
\end{aligned}
\end{align*}
\bottomhrule
\caption[ ]{Preimage arrow lifts needed to interpret probabilistic programs.}
\label{fig:extra-preimage-arrow-defs}
\end{figure*}

\begin{figure*}[!tb]\centering
\smallmathfont
\subfloat[Definitions for preimage mappings that compute rectangular covers.]{
\begin{minipage}{0.98\textwidth}
\begin{align*}
\!\!\begin{aligned}[t]
	&\begin{aligned}[t]
		&X \prepto' Y ::= \pair{Rect~Y, Rect~Y \tto Rect~X}
	\end{aligned} \\
\\[-6pt]
	&\begin{aligned}[t]
		&\emptyset\pre' \ := \ \pair{\emptyset,\fun{B} \emptyset}
	\end{aligned} \\
\\[-6pt]
	&\begin{aligned}[t]
		&ap\pre' : (X \prepto' Y) \tto Rect~Y \tto Rect~X \\
		&ap\pre'~\pair{Y',p}~B := p~(B \i Y') 
	\end{aligned} \\
\\[-6pt]
	&\begin{aligned}[t]
		&(\circ\pre') : (Y \prepto' Z) \tto (X \prepto' Y) \tto (X \prepto' Z) \\
		&\pair{Z',p_2} \circ\pre' h_1 := \pair{Z', \fun{C}{ap\pre'~h_1~(p_2~C)}}
	\end{aligned} \\
\end{aligned}
&\tab\ 
\begin{aligned}[t]
	&\begin{aligned}[t]
		&\pair{\cdot,\cdot}\pre' : (X \prepto' Y_1) \tto (X \prepto' Y_2) \tto (X \prepto' Y_1 \times Y_2) \\
		&\pair{\pair{Y_1',p_1},\pair{Y_2',p_2}}\pre' \ := \\
		&\tab\pair{Y_1' \times Y_2',\fun{B}{p_1~(proj_1~B) \i p_2~(proj_2~B)}}
	\end{aligned} \\
\\[-6pt]
	&\begin{aligned}[t]
		&(\uplus\pre') : (X \prepto' Y) \tto (X \prepto' Y) \tto (X \prepto' Y) \\
		&\pair{Y_1',p_1} \uplus\pre' \pair{Y_2',p_2} \ := \\
		&\tab\pair{Y_1' \join Y_2',\fun{B}{ap\pre'~\pair{Y_1',p_1}~B \join ap\pre'~\pair{Y_2',p_2}~B}
		}
	\end{aligned}
\end{aligned}
\end{align*}
\vspace{3pt}
\hrule
\end{minipage}
\label{fig:approximating-preimage-mapping-defs}
}

\subfloat[Approximating preimage arrow, defined using approximating preimage mappings.]{
\begin{minipage}{0.98\textwidth}
\begin{align*}
\\[-6pt]
\begin{aligned}[t]
	&\begin{aligned}[t]
		&X \preto' Y ::= Rect~X \tto (X \prepto' Y)
	\end{aligned} \\
\\[-6pt]
	&\begin{aligned}[t]
		&(\comppre') : (X \preto' Y) \tto (Y \preto' Z) \tto (X \preto' Z) \\
		&(h_1~\comppre'~h_2)~A \ := \ 
			\lzfclet{
				h_1' & h_1~A \\
				h_2' & h_2~(range\pre'~h_1')
			}{h_2' \circ\pre' h_1'}
	\end{aligned} \\
\\[-6pt]
	&\begin{aligned}[t]
		&(\pairpre') : (X \preto' Y_1) \tto (X \preto' Y_2) \tto (X \preto' \pair{Y_1,Y_2}) \\
		&(h_1~\pairpre'~h_2)~A \ := \ \pair{h_1~A,h_2~A}\pre'
	\end{aligned}
\end{aligned}
&\tab\tab\tab
\begin{aligned}[t]
	&\begin{aligned}[t]
		&\ifpre' : \lzfcsplit{&(X \preto' Bool) \tto (X \preto' Y) \tto \\ &(X \preto' Y) \tto (X \preto' Y)} \\
		&\lzfcsplit{&\ifpre'~h_1~h_2~h_3~A \ := \ \\
			&\tab\lzfclet{
				h_1' & h_1~A \\
				h_2' & h_2~(ap\pre'~h_1'~\set{true}) \\
				h_3' & h_3~(ap\pre'~h_1'~\set{false})
			}{h_2' \uplus\pre' h_3'}}
	\end{aligned} \\
\\[-6pt]
	&\begin{aligned}[t]
		&\lazypre' : (1 \tto (X \preto' Y)) \tto (X \preto' Y) \\
		&\lazypre'~h~A \ := \ if~(A = \emptyset)~\emptyset\pre'~(h~0~A)
	\end{aligned}
\end{aligned}
\end{align*}
\vspace{3pt}
\hrule
\end{minipage}
\label{fig:approximating-preimage-arrow-defs}
}

\subfloat[Preimage* arrow combinators for probabilistic choice and guaranteed termination.
\figref{fig:astore-arrow-defs}~($AStore$ arrow transformer) defines $\arrowtrans\ppre'$, $(\compppre')$, $(\pairppre')$, $\ifppre'$ and $\lazyppre'$.]{
\begin{minipage}{0.98\textwidth}
\begin{align*}
\\[-6pt]
\!\!\!\begin{aligned}[t]
	&\begin{aligned}[t]
		&X \ppreto' Y ::= AStore~(R \times T)~(X \preto' Y)
	\end{aligned} \\
\\[-6pt]
	&\begin{aligned}[t]
		&random\ppre' : X \ppreto' [0,1] \\
		&\lzfcsplit{&random\ppre'~j \ := \ \\ &\tab fst\pre~\comppre'~fst\pre~\comppre'~\pi\pre~j}
	\end{aligned} \\
\\[-6pt]
	&\begin{aligned}[t]
		&branch\ppre' : X \ppreto' Bool \\
		&\lzfcsplit{&branch\ppre'~j \ := \ \\ &\tab fst\pre~\comppre'~snd\pre~\comppre'~\pi\pre~j}
	\end{aligned} \\
\\[-6pt]
	&\begin{aligned}[t]
		fst\ppre' &:= \arrowtrans\ppre'~fst\pre;\ \cdots
	\end{aligned}
\end{aligned}
&
\begin{aligned}[t]
	&\begin{aligned}[t]
		&{\arrowconvif}\ppre' : (X \ppreto' Bool) \tto (X \ppreto' Y) \tto (X \ppreto' Y) \tto (X \ppreto' Y) \!\!\!\!\!\!\!\!\!\\
		&\lzfcsplit{
			&{\arrowconvif}\ppre'~k_1~k_2~k_3~j \ := \\
			&\tab\lzfclet{
				\pair{C_k,p_k} & k_1~(left~j)~A \\
				\pair{C_b,p_b} & branch\ppre~j~A \\
				C_2 & C_k \i C_b \i \set{true} \\
				C_3 & C_k \i C_b \i \set{false} \\
				A_2 & p_k~C_2 \i p_b~C_2 \\
				A_3 & p_k~C_3 \i p_b~C_3 \\
			}{if~\lzfcsplit{
					&(C_b = \set{true,false}) \\
					&\pair{\top,\fun{B}\lzfcsplit{A_2 \join A_3}} \\
					&(k_2~(left~(right~j))~A_2 \uplus\pre' k_3~(right~(right~j))~A_3)\!\!\!\!\!\!\!\!\!}}
		}
	\end{aligned}
\end{aligned}
\end{align*}
\vspace{3pt}
\hrule
\end{minipage}
\label{fig:approximating-preimage*-arrow-defs}
}
\caption[ ]{Implementable arrows that approximate preimage arrows.
Specific lifts such as $fst\pre := \arrpre~fst$ are computable (see \figref{fig:extra-preimage-arrow-defs}), but $\arrpre'$ is not.
}
\label{fig:approximating-arrow-defs}
\end{figure*}


%\appendix
%\section{Appendix Title}

%This is the text of the appendix, if you need one.

%\acks

%Acknowledgments, if needed.


\mathversion{normal}

% We recommend abbrvnat bibliography style.

\bibliographystyle{abbrvnat}

% The bibliography should be embedded for final submission.

\softraggedright
\bibliography{local-cites}

\end{document}
