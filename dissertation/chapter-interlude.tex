Now that we are satisfied that using \lzfclang as a target language for categorical semantics of constructive theories and queries works, we turn our attention to uncountable sample spaces and theories with general recursion.

It seems that, having followed measure-theoretic structure so far, the extension to uncountable sample spaces should be fairly smooth.
Discrete probability spaces, probability mass functions, summation, conditioning, and discrete transition kernels all have uncountable analogues that can be composed in much the same way.
The probability space monad thus has an uncountable analogue that can be used as a target for a categorial semantics for Bayesian notation.
There are two difficulties, however.

The first difficulty is practical.
As with the discrete probability space monad, we would like to \emph{derive} an implementable semantics by approximating the target category.
Unfortunately, this is complicated by the fact that the uncountable computations have large cardinalities.
For example, a general probability space on $\Re$ is defined as a triple $\pair{\Re,\Sigma,P}$, where $\Sigma$ is a subset of $\powerset~\Re$ and $P : \Sigma \to [0,1]$.

The second difficulty is theoretical.
Suppose we define the following recursive function in a language with probabilistic choice, which counts the number of times $random < p$:
\begin{equation}
	geometric~p \ := \ if~(random < p)~0~(1 + geometric~p)
\end{equation}
To interpret $geometric~p$ using the uncountable probability space monad, we must interpret both branches of the $if$ as probability spaces and merge them.
Unfortunately, doing so na\"ively results in nontermination, as $geometric~p$ is applied in the \emph{else} branch at every recurrence.
Dealing with nontermination requires complicated fixpoint constructions and limits, which, with uncountable probability spaces, would put us on the frontier of research in mathematics instead of in computer science.

Fortunately, we can take a hint from measure-theoretic probability's general approach to infinite processes: define them with respect to a canonical, infinite-dimensional probability space, and encode branching and other complexities into the random variables.
The next chapter takes this approach by interpreting whole programs as random variables in which every $random$ expression indexes an infinite, random tree.
