%\usepackage{showframe}
\usepackage{lmodern}
\usepackage[T1]{fontenc}
\usepackage{xspace}
\usepackage{graphicx}
\usepackage{slatex}
\usepackage{subfig}
\usepackage{varwidth}
\usepackage{verbatim}
\usepackage{color}
\usepackage{stmaryrd}
\usepackage{amsmath}
\usepackage{amssymb}
\usepackage{amsthm}
\usepackage{amscd}
\usepackage{bbm}
\usepackage{mathtools}
\usepackage{caption}

\renewcommand{\captionfont}{\footnotesize}

\usepackage{xparse}
\makeatletter
\NewDocumentCommand{\raisedminus}{m}{%
  \raisebox{0.2em}{$\m@th#1{-}$}%
}
\NewDocumentCommand{\unaryminus}{}{%
  \mathbin{%
    \mathchoice{%
      \raisedminus\scriptstyle
    }{%
      \raisedminus\scriptstyle
    }{%
      \raisedminus\scriptscriptstyle
    }{%
      \raisedminus\scriptscriptstyle
    }%
  }%
}
\makeatother

% Hyperref likes to go last
\usepackage[
    bookmarks=true,
    bookmarksnumbered=true,
    breaklinks=false,
    raiselinks=true,
    pdfborder={0 0 0},
    colorlinks=false,
    plainpages=false,
    ]{hyperref}

% This makes hyperlinks point to the tops of figures, not their captions
\usepackage[all]{hypcap}

%%%%%%%%%%%%%%%%%%%%%%%%%%%%%%%%%%%%%%%%%%%%%%%%%%%%%%%%%%%%%%%%%%%%%%%%%%%%%%%%%%%%%%%%%%%%%%%%%%%%%
% BEGIN SPACE HACKS
%%%%%%%%%%%%%%%%%%%%%%%%%%%%%%%%%%%%%%%%%%%%%%%%%%%%%%%%%%%%%%%%%%%%%%%%%%%%%%%%%%%%%%%%%%%%%%%%%%%%%

\makeatletter
\renewcommand\normalsize{%
   \@setfontsize\normalsize\@@ptsize{14.5pt}%
   \abovedisplayskip 8\p@ \@plus2\p@ \@minus2\p@
   \belowdisplayskip \abovedisplayskip
   \abovedisplayshortskip 4\p@ \@plus2\p@ \@minus2\p@
   \belowdisplayshortskip \abovedisplayshortskip
   \renewcommand\arraystretch{0.65}
}

\renewcommand\small{%
   \@setfontsize\small\@ixpt{12pt}%
   \abovedisplayskip 7\p@ \@plus2\p@ \@minus2\p@
   \belowdisplayskip \abovedisplayskip
   %\abovedisplayshortskip 0\p@
   %\belowdisplayshortskip 0\p@
   %\belowdisplayskip \abovedisplayskip
   \renewcommand\arraystretch{0.9}
}

%\setlength\floatsep       {3pt \@minus 1\p@}
%\setlength\textfloatsep   {12pt \@minus 4\p@}
%\setlength\intextsep      {3pt \@minus 1\p@}
%\setlength\dblfloatsep    {3pt \@minus 1\p@}
%\setlength\dbltextfloatsep{3pt \@minus 1\p@}
\makeatother

\usepackage{enumitem}
\setlist{noitemsep,topsep=0pt,leftmargin=36pt}
\setitemize{noitemsep,topsep=0pt}
\setenumerate{noitemsep,topsep=0pt}

\newcommand{\smallmathfont}{\small}

\makeatletter
     \renewcommand*\l@figure{\@dottedtocline{1}{1em}{3.2em}}
\makeatother

%%%%%%%%%%%%%%%%%%%%%%%%%%%%%%%%%%%%%%%%%%%%%%%%%%%%%%%%%%%%%%%%%%%%%%%%%%%%%%%%%%%%%%%%%%%%%%%%%%%%%
% END SPACE HACKS
%%%%%%%%%%%%%%%%%%%%%%%%%%%%%%%%%%%%%%%%%%%%%%%%%%%%%%%%%%%%%%%%%%%%%%%%%%%%%%%%%%%%%%%%%%%%%%%%%%%%%

% Hide subsections in TOC
%\setcounter{tocdepth}{1}

\newcounter{maththing}\numberwithin{section}{chapter}

\theoremstyle{definition}
\newtheorem{example}{Example}[chapter]

\theoremstyle{plain}
\newtheorem{definition}[example]{Definition}
\newtheorem{theorem}[example]{Theorem}
\newtheorem{lemma}[example]{Lemma}
\newtheorem{corollary}[example]{Corollary}

\newcommand\numberthis{\addtocounter{equation}{1}\tag{\theequation}}

% puts a little space between the \hrule and captions
\belowcaptionskip 0.25\baselineskip
\newcommand{\bottomhrule}{\vspace{3pt}\hrule\vspace{3pt}}

% keywords
\newcommand{\keyword}[1]{\textbf{#1}}
\newcommand{\mykeyword}[1]{\textbf{\textit{#1}}}

\newcommand\xqed[1]{%
  \leavevmode\unskip\penalty9999 \hbox{}\nobreak\hfill
  \quad\hbox{#1}}

\newcommand{\exampleqed}{\xqed{$\diamondsuit$}}


\newcommand{\lzfclang}{\ensuremath{\lambda_{\text{ZFC}}}\xspace}
\newcommand{\targetlang}{\lzfclang}

%
% For IFL paper
%

\newcommand{\uscore}{\underbar{ }\underbar{ }}

% application syntax
\newcommand{\app}[2]{{#1}\ {#2}}
\newcommand{\appp}[3]{{#1}\ {#2}\ {#3}}
\newcommand{\apppp}[4]{{#1}\ {#2}\ {#3}\ {#4}}
\newcommand{\appppp}[5]{{#1}\ {#2}\ {#3}\ {#4}\ {#5}}
\newcommand{\apppppp}[6]{{#1}\ {#2}\ {#3}\ {#4}\ {#5}\ {#6}}
% parenthesized application syntax
\newcommand{\papp}[2]{\left({#1}\ {#2}\right)}
\newcommand{\pappp}[3]{\left({#1}\ {#2}\ {#3}\right)}
\newcommand{\papppp}[4]{\left({#1}\ {#2}\ {#3}\ {#4}\right)}
\newcommand{\pappppp}[5]{\left({#1}\ {#2}\ {#3}\ {#4}\ {#5}\right)}
\newcommand{\papppppp}[6]{\left({#1}\ {#2}\ {#3}\ {#4}\ {#5}\ {#6}\right)}

% boolean stuff
\newcommand{\true}{\mathit{true}}
\newcommand{\false}{\mathit{false}}

\newcommand{\where}[1]{
	\begin{aligned}
		\;\;\text{where}\ \ #1
	\end{aligned}
}

\newcommand{\tto}{\Rightarrow}

\newcommand{\even}{\mathsf{even}}
\newcommand{\odd}{\mathsf{odd}}
\newcommand{\low}{\mathsf{low}}
\newcommand{\high}{\mathsf{high}}

\newcommand{\purerv}{pure_{rv}}
\newcommand{\applyrv}{ap^*_{rv}}
\newcommand{\RV}[1]{\mathcal{R}\!\meaningof{#1}}

\newcommand{\prob}{sum}
\newcommand{\dist}{dist}
\newcommand{\finitize}{finitize}
\newcommand{\Prob}[1]{\mathbf{P}\!\meaningof{#1}}
\newcommand{\Dist}[1]{\mathbf{D}\!\meaningof{#1}}

\newcommand{\given}{\,|\,}
\newcommand{\cond}{cond}

\newcommand{\ret}{return_{ps}}
\newcommand{\bind}{bind_{ps}}

\newcommand{\extendps}{extend_{ps}}
\newcommand{\condps}{cond_{ps}}
\newcommand{\distps}{dist_{ps}}
\newcommand{\runps}{run_{ps}}

\newcommand{\retset}{return_{set}}
\newcommand{\bindset}{bind_{set}}

\newcommand{\PS}[1]{\mathcal{M}\!\meaningof{#1}}

\newcommand{\objlang}[1]{\mathit{#1}}
\newcommand{\Pmeas}{\mathbb{P}}
\newcommand{\Pspec}[1]{\mathrm{Pr}\!\left[{#1}\right]}

\newcommand{\restrict}[2]{{#1}\big|_{#2}}

\newcommand{\powerset}{\mathcal{P}}
\newcommand{\domain}{domain}
\newcommand{\codomain}{codomain}
\newcommand{\range}{range}
\newcommand{\image}{image}
\newcommand{\preimage}{preimage}
\newcommand{\fst}{fst}
\newcommand{\snd}{snd}

\newcommand{\equivdist}{\equiv_\mathbf{D}}

%
% Language phase 0: meta
%
% Mostly semantic functions and grammars, definitional extensions to first-order logic

\newcommand{\metadef}{:=}
\newcommand{\metastx}{:\equiv}
\newcommand{\objdef}{:=}
\newcommand{\objstx}{:\equiv}
\newcommand{\gor}{\ |\ }
\newcommand{\meaningof}[1]{\left\llbracket{#1}\right\rrbracket}
\newcommand{\enc}[1]{\mathcal{F}\!\meaningof{#1}}
\newcommand{\stx}[1]{\mathcal{S}\!\meaningof{#1}}
%\newcommand{\metasubst}[3]{{#1}[{#2}:={#3}]}
\newcommand{\metasubst}[3]{s\!\meaningof{{#1},{#2},{#3}}}
\newcommand{\imp}{\Rightarrow}
\newcommand{\rimp}{\Leftarrow}
\DeclareMathOperator{\disjoint}{\mbox{$\not\mspace{-5mu}\i$}}

\renewcommand{\dots}{...}
\newcommand{\setb}[2]{\lbrace {#1} \ \lvert\ {#2} \rbrace}

\newcommand{\justdenom}[1]{\mspace{-4mu} {} \atop {#1} \mspace{-4mu}}
\newcommand{\djustdenom}[1]{\displaystyle{\justdenom{#1}}}


%
% Language phase 1: first-order logic
%

% Common notation and names

\newcommand{\set}[1]{\{{#1}\}}
\newcommand{\pair}[1]{\langle{#1}\rangle}
\newcommand{\dpair}[1]{\left\langle{#1}\right\rangle}
\newcommand{\seq}{\pair}

\newcommand{\band}{\wedge}
\newcommand{\pand}{\big\wedge}
\newcommand{\bor}{\vee}
\newcommand{\por}{\big\vee}

\newcommand{\Bool}{\mathbb{B}}
\newcommand{\Nat}{\mathbb{N}}
\newcommand{\Int}{\mathbb{Z}}
\newcommand{\Rat}{\mathbb{Q}}
\renewcommand{\Re}{\mathbb{R}}
\renewcommand{\P}{\mathbb{P}}

\newcommand{\U}{\textstyle\bigcup}
\renewcommand{\u}{\cup}
\newcommand{\I}{\textstyle\bigcap}
\renewcommand{\i}{\cap}
\newcommand{\wo}{\backslash}
\let\oldemptyset\emptyset
\renewcommand{\emptyset}{\varnothing}

\newcommand{\Forall}[1]{\forall\,{#1}\ldotp\,}
\newcommand{\Exists}[1]{\exists\,{#1}\ldotp\,}
\newcommand{\ExistsOne}[1]{\exists\char`^\,{#1}\ldotp\,}
\newcommand{\The}[1]{\iota\,{#1}\ldotp\,}

%\newcommand{\bijto}{\xrightarrow{\mathrm{bij}}}
\newcommand{\bijto}{\leftrightarrow}
\newcommand{\Ord}{\mathit{Ord}}

\newcommand{\A}{\mathcal{A}}
\newcommand{\B}{\mathcal{B}}
\newcommand{\C}{\mathcal{C}}
\newcommand{\E}{\mathcal{E}}
\newcommand{\V}{\mathcal{V}}

\newcommand{\pnand}{\mathit{nand}}
\newcommand{\w}{\backslash}

% Invented notation and names

\newcommand{\myfun}[1]{\mathit{#1}}
\newcommand{\encset}{\myfun{set}}
\newcommand{\lex}{\myfun{lex}}
\newcommand{\addset}{\myfun{put}}

%\newcommand{\dto}{\Downarrow}
% makes the horizontal spacing around the arrow nice:
\DeclareMathOperator{\dto}{\Downarrow}

\newcommand{\plusomega}{+_\lnat}
\newcommand{\timesomega}{\times_\lnat}

\newcommand{\plusint}{+_Z}
\newcommand{\minusint}{-_Z}
\newcommand{\timesint}{\times_Z}
\newcommand{\equalint}{=_{\mathsf{Z}}}

\newcommand{\plusz}{+_\Int}
\newcommand{\minusz}{-_\Int}
\newcommand{\timesz}{\times_\Int}

\newcommand{\equalrat}{=_{\mathsf{Q}}}
\newcommand{\plusq}{+_\Rat}
\newcommand{\minusq}{-_\Rat}
\newcommand{\timesq}{\times_\Rat}
\newcommand{\divq}{\div_\Rat}

%
% Language phase 2: lambda_ZFC
%

\newcommand{\mathobjlang}[1]{\mathsf{#1}}
%\newcommand{\mathobjlang}[1]{\mathtt{#1}}

\renewcommand{\choose}{\mathcal{E}}
%\newcommand{\Choose}[1]{\choose\,{#1}\ldotp\,}
\newcommand{\Choose}[1]{\set{#1}\ldotp\,}

\renewcommand{\domain}{\mathobjlang{domain}}
\renewcommand{\range}{\mathobjlang{range}}
\renewcommand{\image}{\mathobjlang{image}}
\newcommand{\filter}{\mathobjlang{select}}

\newcommand{\lunion}[1]{{\textstyle\U}\ {#1}}
\newcommand{\lpowerset}[1]{\powerset\ {#1}}
\newcommand{\limage}[2]{{\mathobjlang{image}}\ {#1}\ {#2}}
\newcommand{\lorder}[1]{{\mathobjlang{card}}\ {#1}}
\newcommand{\ltake}[1]{{\mathobjlang{take}}\ {#1}}
%\newcommand{\lemptyset}{\oldemptyset}
\newcommand{\lemptyset}{\emptyset}
\newcommand{\lnat}{\omega}
\newcommand{\lif}[3]{{\mathobjlang{if}}\ {#1}\ {#2}\ {#3}}
\newcommand{\ltrue}{\mathobjlang{true}}
\newcommand{\lfalse}{\mathobjlang{false}}
\newcommand{\lfun}{\lambda\ldotp\,}

\newcommand{\fun}[1]{\lambda\mspace{2mu}{#1}\ldotp\,}

\newcommand{\tvar}{t_\mathrm{var}}
\newcommand{\tapp}{t_\mathrm{app}}
\newcommand{\tif}{t_\mathrm{if}}
\newcommand{\tin}{t_\in}
\newcommand{\tunion}{t_\u}
\newcommand{\ttake}{t_\mathrm{take}}
\newcommand{\tpowerset}{t_\powerset}
\newcommand{\timage}{t_\mathrm{image}}
\newcommand{\torder}{t_\mathrm{card}}
\newcommand{\tset}{t_\mathrm{set}}
\newcommand{\tatom}{t_\mathrm{atom}}
\newcommand{\tfun}{t_\lambda}
\newcommand{\atrue}{a_\mathrm{true}}
\newcommand{\afalse}{a_\mathrm{false}}
\newcommand{\ttrue}{t_\mathrm{true}}
\newcommand{\tfalse}{t_\mathrm{false}}

\newcommand{\Vatom}{V_\mathrm{atom}}
\newcommand{\Vset}{V_\mathrm{set}}
\newcommand{\Vfun}{V_\lambda}
\newcommand{\Evar}{E_\mathrm{var}}
\newcommand{\Eapp}{E_\mathrm{app}}
\newcommand{\Eset}{E_\mathrm{set}}
\newcommand{\Eif}{E_\mathrm{if}}
\newcommand{\Ein}{E_\in}
\newcommand{\Eunion}{E_\u}
\newcommand{\Epowerset}{E_\powerset}
\newcommand{\Eimage}{E_\mathrm{image}}
\newcommand{\Eorder}{E_\mathrm{card}}

\renewcommand{\fst}{\myfun{fst}}
\renewcommand{\snd}{\myfun{snd}}

\newcommand{\jand}{\;\;\,}

\newenvironment{displaybreaks}%
{%
	\begingroup%
	\allowdisplaybreaks%
}%
{%
	\endgroup%
	\ignorespacesafterend%
}

%\excludecomment{proof}

\newcommand{\figref}[1]{Figure~\ref{#1}}
\newcommand{\figsref}[1]{Figures~\ref{#1}}

\newcommand{\arrow}{\rightsquigarrow}

\renewcommand{\restrict}[1]{\lvert_{#1}}
\newcommand{\pto}{\rightharpoonup}
\newcommand{\Un}{\mathcal{U}}

\newcommand{\join}{\vee}

\newcommand{\conv}{^{\mspace{-2mu}\Downarrow\mspace{-2mu}}}

\newcommand{\meaningofconv}[1]{\left\llbracket{#1}\right\rrbracket\conv}

\newcommand{\arrowlift}{\ensuremath{lift}}
\newcommand{\arrowarr}{\ensuremath{arr}}
\newcommand{\arrowcomp}{\ensuremath{{>}\mspace{-6mu}{>}\mspace{-6mu}{>}}}
\newcommand{\arrowpair}{\ensuremath{\mathit{\&\mspace{-7.5mu}\&\mspace{-7.5mu}\&}}}
\newcommand{\arrowif}{\ensuremath{ifte}}
\newcommand{\arrowconvif}{\ensuremath{ifte\conv}}
\newcommand{\arrowlazy}{\ensuremath{lazy}}
\newcommand{\arrowapp}{\ensuremath{app}}
\newcommand{\arrowrun}{\ensuremath{run}}
\newcommand{\arrowget}{\ensuremath{get}}
\newcommand{\arrowerror}{\ensuremath{error}}
\newcommand{\arrowtrans}{\ensuremath{\eta}}

\newcommand{\gen}{_\mathrm{a}}
\newcommand{\genb}{_\mathrm{b}}
\newcommand{\genc}{_\mathrm{a^{\mspace{-2mu}*}}}
\newcommand{\gend}{_\mathrm{b^{\mspace{-2mu}*}}}

\DeclareMathOperator{\botto}{\arrow_{\mspace{-3mu}\bot}}
\newcommand{\arrbot}{\arrowarr_{\mspace{-3mu}\bot}}
\newcommand{\compbot}{\arrowcomp_{\mspace{-5mu}\bot}}
\newcommand{\pairbot}{\arrowpair_{\mspace{-3mu}\bot}}
\newcommand{\ifbot}{\arrowif_{\mspace{-2mu}\bot}}
\newcommand{\lazybot}{\arrowlazy_{\mspace{-2mu}\bot}}

\newcommand{\map}{_\mathrm{map}}
\DeclareMathOperator{\mapto}{\arrow_{\mspace{-23mu}\map}}
\newcommand{\liftmap}{\arrowlift\map}
\newcommand{\arrmap}{\arrowarr\map}
\newcommand{\compmap}{\arrowcomp\map}
\newcommand{\pairmap}{\arrowpair\map}
\newcommand{\ifmap}{\arrowif\map}
\newcommand{\lazymap}{\arrowlazy\map}

\newcommand{\pre}{_\mathrm{pre}}
\DeclareMathOperator{\preto}{\arrow_{\mspace{-20mu}\pre}}
\newcommand{\liftpre}{\arrowlift\pre}
\newcommand{\arrpre}{\arrowarr\pre}
\newcommand{\comppre}{\arrowcomp\pre}
\newcommand{\pairpre}{\arrowpair\pre}
\newcommand{\ifpre}{\arrowif\pre}
\newcommand{\lazypre}{\arrowlazy\pre}

\newcommand{\pbot}{{\bot^{\mspace{-4mu}*}}}
\DeclareMathOperator{\pbotto}{\arrow_{\mspace{-3mu}\pbot}}
\newcommand{\arrpbot}{\arrowarr_{\mspace{-3mu}\pbot}}
\newcommand{\comppbot}{\arrowcomp_{\mspace{-5mu}\pbot}}
\newcommand{\pairpbot}{\arrowpair_{\mspace{-3mu}\pbot}}
\newcommand{\ifpbot}{\arrowif_{\mspace{-2mu}\pbot}}
\newcommand{\convifpbot}{\arrowconvif_{\mspace{-2mu}\pbot}}
\newcommand{\lazypbot}{\arrowlazy_{\mspace{-2mu}\pbot}}

\newcommand{\pmap}{_\mathrm{map^{\mspace{-2mu}*}}}
\DeclareMathOperator{\pmapto}{\arrow_{\mspace{-25mu}_{\mathrm{map*}}}}
\newcommand{\liftpmap}{\arrowlift\pmap}
\newcommand{\arrpmap}{\arrowarr\pmap}
\newcommand{\comppmap}{\arrowcomp\pmap}
\newcommand{\pairpmap}{\arrowpair\pmap}
\newcommand{\ifpmap}{\arrowif\pmap}
\newcommand{\convifpmap}{\arrowconvif\pmap}
\newcommand{\lazypmap}{\arrowlazy\pmap}

\newcommand{\ppre}{_\mathrm{pre^{\mspace{-2mu}*}}}
\DeclareMathOperator{\ppreto}{\arrow_{\mspace{-21mu}_{\mathrm{pre*}}}}
\newcommand{\liftppre}{\arrowlift\ppre}
\newcommand{\arrppre}{\arrowarr\ppre}
\newcommand{\compppre}{\arrowcomp\ppre}
\newcommand{\pairppre}{\arrowpair\ppre}
\newcommand{\ifppre}{\arrowif\ppre}
\newcommand{\convifppre}{\arrowconvif\ppre}
\newcommand{\lazyppre}{\arrowlazy\ppre}

\newcommand{\prepto}{\pto_{\mspace{-19mu}\pre}}

\newcommand{\pbi}{_\mathrm{idxs^{\mspace{-2mu}*}}}
\DeclareMathOperator{\pbito}{\arrow_{\mspace{-19mu}_{\mathrm{idxs*}}}}


\DeclareMathVersion{sans}

\SetSymbolFont{operators}{sans}{T1}{\sfdefault}{m}{n}
\SetSymbolFont{letters}{sans}{T1}{\sfdefault}{m}{n}
\SetMathAlphabet\mathbf{sans}{T1}{\sfdefault}{bx}{n}
\SetMathAlphabet\mathsf{sans}{T1}{\sfdefault}{m}{n}
\SetMathAlphabet\mathit{sans}{T1}{\rmdefault}{m}{it}
%\SetMathAlphabet\mathrm{sans}{T1}{\rmdefault}{m}{n}

% Fix the uppercase I in sans-serif so it doesn't look like a lowercase L
\DeclareSymbolFont{rmletters}{T1}{\rmdefault}{m}{it}
\SetSymbolFont{rmletters}{sans}{T1}{\rmdefault}{m}{n}
\DeclareMathSymbol{I}{\mathalpha}{rmletters}{`I}

% Fix the uppercase greek letters in sans-serif
\DeclareSymbolFont{greekletters}{OML}{cmss}{m}{n}
\DeclareMathSymbol{\Gamma}  {\mathord}{greekletters}{"00}
\DeclareMathSymbol{\Delta}  {\mathord}{greekletters}{"01}
\DeclareMathSymbol{\Theta}  {\mathord}{greekletters}{"02}
\DeclareMathSymbol{\Lambda} {\mathord}{greekletters}{"03}
\DeclareMathSymbol{\Xi}     {\mathord}{greekletters}{"04}
\DeclareMathSymbol{\Pi}     {\mathord}{greekletters}{"05}
\DeclareMathSymbol{\Sigma}  {\mathord}{greekletters}{"06}
\DeclareMathSymbol{\Upsilon}{\mathord}{greekletters}{"07}
\DeclareMathSymbol{\Phi}    {\mathord}{greekletters}{"08}
\DeclareMathSymbol{\Psi}    {\mathord}{greekletters}{"09}
\DeclareMathSymbol{\Omega}  {\mathord}{greekletters}{"0A}

% Fix the lowercase greek letters in sans-serif
\DeclareMathSymbol{\alpha}{\mathord}{greekletters}{"0B}
\DeclareMathSymbol{\beta}{\mathord}{greekletters}{"0C}
\DeclareMathSymbol{\gamma}{\mathord}{greekletters}{"0D}
\DeclareMathSymbol{\delta}{\mathord}{greekletters}{"0E}
\DeclareMathSymbol{\epsilon}{\mathord}{greekletters}{"0F}
\DeclareMathSymbol{\zeta}{\mathord}{greekletters}{"10}
\DeclareMathSymbol{\eta}{\mathord}{greekletters}{"11}
\DeclareMathSymbol{\theta}{\mathord}{greekletters}{"12}
\DeclareMathSymbol{\iota}{\mathord}{greekletters}{"13}
\DeclareMathSymbol{\kappa}{\mathord}{greekletters}{"14}
\DeclareMathSymbol{\lambda}{\mathord}{greekletters}{"15}
\DeclareMathSymbol{\mu}{\mathord}{greekletters}{"16}
\DeclareMathSymbol{\nu}{\mathord}{greekletters}{"17}
\DeclareMathSymbol{\xi}{\mathord}{greekletters}{"18}
\DeclareMathSymbol{\pi}{\mathord}{greekletters}{"19}
\DeclareMathSymbol{\rho}{\mathord}{greekletters}{"1A}
\DeclareMathSymbol{\sigma}{\mathord}{greekletters}{"1B}
\DeclareMathSymbol{\tau}{\mathord}{greekletters}{"1C}
\DeclareMathSymbol{\upsilon}{\mathord}{greekletters}{"1D}
\DeclareMathSymbol{\phi}{\mathord}{greekletters}{"1E}
\DeclareMathSymbol{\chi}{\mathord}{greekletters}{"1F}
\DeclareMathSymbol{\psi}{\mathord}{greekletters}{"20}
\DeclareMathSymbol{\omega}{\mathord}{greekletters}{"21}
\DeclareMathSymbol{\varepsilon}{\mathord}{greekletters}{"22}
\DeclareMathSymbol{\vartheta}{\mathord}{greekletters}{"23}
\DeclareMathSymbol{\varpi}{\mathord}{greekletters}{"24}
\DeclareMathSymbol{\varrho}{\mathord}{greekletters}{"25}
\DeclareMathSymbol{\varsigma}{\mathord}{greekletters}{"26}

% Fix the punctuation everywhere else
\DeclareSymbolFont{fixpunct}{T1}{\rmdefault}{m}{n}
\DeclareSymbolFont{fixpunctoml}{OML}{\rmdefault}{m}{n}
\DeclareMathSymbol{,}{\mathpunct}{fixpunct}{`,}
\DeclareMathSymbol{.}{\mathord}{fixpunct}{`.}
\DeclareMathSymbol{\ldotp}{\mathord}{fixpunct}{`.}
\DeclareMathSymbol{!}{\mathord}{letters}{`-}
\DeclareMathSymbol{/}{\mathpunct}{fixpunct}{`/}
\DeclareMathSymbol{\rightharpoonup}{\mathrel}{fixpunctoml}{"2A}
\DeclareMathSymbol{^}{\mathpunct}{fixpunct}{`!}

% Make numbers always roman, never italicized
\DeclareSymbolFont{numbers}{T1}{pvh}{m}{n}
%\SetSymbolFont{numbers}{sans}{T1}{\rmdefault}{m}{n}
\DeclareMathSymbol{0}{\mathord}{numbers}{"30}
\DeclareMathSymbol{1}{\mathord}{numbers}{"31}
\DeclareMathSymbol{2}{\mathord}{numbers}{"32}
\DeclareMathSymbol{3}{\mathord}{numbers}{"33}
\DeclareMathSymbol{4}{\mathord}{numbers}{"34}
\DeclareMathSymbol{5}{\mathord}{numbers}{"35}
\DeclareMathSymbol{6}{\mathord}{numbers}{"36}
\DeclareMathSymbol{7}{\mathord}{numbers}{"37}
\DeclareMathSymbol{8}{\mathord}{numbers}{"38}
\DeclareMathSymbol{9}{\mathord}{numbers}{"39}

\renewcommand{\notin}{\not\in}

\newcommand\mathtest[2]{\mathchoice{#1}{#2}{#2}{#2}}

\newenvironment{lzfcenv}[1][l]%
{%
	\begin{array}[t]{@{}#1@{}}%
}%
{%
	\end{array}%
}

\newcommand{\tab}{\ \ \ }

\newcommand{\lzfc}[2][l]{\begin{lzfcenv}[#1]#2\end{lzfcenv}}
\newcommand{\tlzfc}[1]{\ensuremath{#1}}

\newcommand{\lzfcsplit}[2][@{}]{\lzfc[r #1 l]{#2}}

\newcommand{\lzfclet}[2]{\lzfc{let\ \ \lzfc[r@{\ \objdef\ }l]{#1} \\ \lzfc{\tab in\ \ #2}}}
\newcommand{\lzfccond}[2][\ \longrightarrow\ ]{\lzfc{cond\ \ \lzfc[l@{\ {#1}\ }l]{#2}}}
\newcommand{\lzfccase}[3][\ \longrightarrow\ ]{\lzfc{case\ \ {#2}\\\ \ \lzfc[l@{\ {#1}\ }l]{#3}}}
\newcommand{\lzfcif}[3]{if~\lzfcsplit{&{#1}\\&{#2}\\&{#3}}}

\newcommand{\dash}{\text{-}}

%
% Haskell syntax
%

\makeatletter
\newcommand{\codesize}{\@setfontsize\small{11.5pt}{13pt}}
\makeatother

\usepackage{minted}
\usemintedstyle{vs}
\newminted{haskell}{%
fontsize=\codesize,%
%frame=single%
}

%
% Racket syntax (must be last - or at least \defschememathescape must be - see last line)
%

% DrRacket's default colors
\definecolor{identifiercolor}{rgb}{0.15,0.15,0.5}
\definecolor{keywordcolor}{rgb}{0.0,0.0,0.0}
\definecolor{constantcolor}{rgb}{0.16,0.5,0.15}
\definecolor{parenthesiscolor}{rgb}{0.52,0.24,0.14}

\def\keywordfont#1{\textcolor{keywordcolor}{#1}}
\def\variablefont#1{\textcolor{identifiercolor}{#1}}
\def\constantfont#1{\textcolor{constantcolor}{#1}}
\def\datafont#1{\textcolor{parenthesiscolor}{#1}}

\def\mycodehook{\tt\color{parenthesiscolor}\codesize}

\let\schemecodehook\mycodehook
\setlength{\leftcodeskip}{\parindent}

\setlength{\leftcodeskip}{0pt}
\setlength{\rightcodeskip}{0pt plus 1fil}
\setlength{\abovecodeskip}{0.5\abovedisplayskip}
\setlength{\belowcodeskip}{0.5\belowdisplayskip}

\unsetspecialsymbol{.}
\unsetspecialsymbol{...}
\unsetspecialsymbol{-}
\unsetspecialsymbol{1-}
\unsetspecialsymbol{-1+}

\setspecialsymbol{<blank-line>}{\va{}}
\setspecialsymbol{lambda}{\va{$\mathtt{\lambda}$}}
\setspecialsymbol{λ}{\va{$\mathtt{\lambda}$}}
\setspecialsymbol{~}{\va{$\mathtt{\sim}$}}
\setspecialsymbol{_}{\va{$\underbar{ }$}}
\setspecialsymbol{Omega}{\va{$\mathtt{\Omega}$}}
\setspecialsymbol{omega}{\va{$\mathtt{\omega}$}}
\setspecialsymbol{Ω}{\va{$\mathtt{\Omega}$}}
\setspecialsymbol{ω}{\va{$\mathtt{\omega}$}}
\setspecialsymbol{mu}{\va{$\mu$}}
\setspecialsymbol{sigma}{\va{$\sigma$}}

\setkeyword{define/drbayes struct/drbayes define-type define-singleton-type U Rec -> : struct: case-> define-predicate match match* ? match-define λ Real Boolean Pair Null Flonum define-values Promise Values match-let}

\defschememathescape{$}
