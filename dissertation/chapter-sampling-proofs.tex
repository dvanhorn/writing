\mathversion{sans}

XXX: You're still here?

\section{Basic Definitions}

XXX: unlike preceeding chapter, basic definitions not meant to be self-contained, but to 1) define things typically regarded as ``just syntax'' or ``just formalism'' as $\lambda$-calculus functions so we can manipulate them more mechanically; 2) give interested readers terminology to look up; 3) give precise definitions for ambiguous terms (especially ``uniformly $\sigma$-finite'')

\begin{definition}[almost-everywhere equality]
\label{def:almost-everywhere-equality}
A measurable predicate $p?$ holds \keyword{almost everywhere} with respect to measure $m$ when it holds on the \emph{complement} of a measure-zero set:
\begin{equation}
\begin{aligned}
	&ae? : (Set~X \pto [0,+\infty]) \tto (X \pto Bool) \tto Bool \\
	&ae?~m~p?\ :=\ m~(preimage~p?~\set{false}) = 0
\end{aligned}
\end{equation}
Similarly, two total mappings are equal almost everywhere (with respect to a measure $m$) when they are equal on all but a measure-zero set:
\begin{equation}
\begin{aligned}
	&ae!equal? : (Set~X \pto [0,+\infty]) \tto (X \to Y) \tto (X \to Y) \tto Bool \\
	&ae!equal?~m~g_1~g_2\ :=\ ae?~m~\fun{a \in domain~g_1} g_1~a = g_2~a
\end{aligned}
\end{equation}
\end{definition}

From here on, we write ``$g_1 = g_2$ ($m$-a.e.)'' instead of $ae!equal?~m~g_1~g_2$.

[XXX: define $\sigma$-finite]

Lebesgue\footnote{Pronounced ``lehBEG,'' and named after French mathematician Henri Lebesgue.} integration:
\begin{equation}
\begin{aligned}
	&int : (X \to \Re) \tto (Set~X \pto [0,+\infty]) \tto (Set~X \pto [-\infty,+\infty]) \\
	&int~f~m~A\ =\ \int_A~f~\mathit{d}m
\end{aligned}
\label{eqn:int-type}
\end{equation}
XXX: $\lzfclang$-definable, but its definition does not give extra clarity

XXX: Lebesgue integration is awesome as a lambda because indefinite integration, which yields a \emph{measure}, can be done simply by partial application; will demonstrate this shortly in the statement of Lemma~\ref{lem:real-almost-everywhere-equality}

% Klenke, Def. 4.13 & Rem. 4.14
\begin{lemma}[indefinite integration yields measures]
If $f : X \to [0,+\infty)$ is measurable and $m : Set~X \pto [0,+\infty]$ is a measure, then $int~f~m$ is a measure.
[XXX: preservation of measure properties like $\sigma$-finite?]
\end{lemma}

For real-valued functions, Lebesgue integration gives another, sometimes more convenient way to characterize almost-everywhere equality: two functions are equal almost everywhere if and only if their indefinite integrals are equal.
\begin{lemma}[real function a.e. equality]
\label{lem:real-almost-everywhere-equality}
If $m : Set~X \pto [0,+\infty]$ is a $\sigma$-finite measure and $f_1,f_2 : X \to \Re$ are measurable, then $f_1 = f_2$ ($m$-a.e) if and only if $int~f_1~m = int~f_2~m$.
\end{lemma}

In differential calculus (hereafter simply ``calculus''), indefinite integration has an inverse: differentiation.
In measure theory, indefinite Lebesgue integration also has an inverse, which is also called differentiation.
In calculus, differentiation is defined only for differentiable functions.
In measure theory, the analogous property is called absolute continuity.

\begin{definition}[absolute continuity]
\label{def:absolute-continuity}
Given measures $m_1,m_2 : Set~X \pto [0,+\infty]$, $m_1$ is \keyword{absolutely continuous} with respect to $m_2$ if $m_1 \ll m_2$, where
\begin{equation}
\begin{aligned}
	&(\ll) : (Set~X \pto [0,+\infty]) \tto (Set~X \pto [0,+\infty]) \tto Bool \\
	&m_1 \ll m_2\ :=\ \Forall{A \in domain~m_2} m_2~A = 0 \implies m_1~A = 0
\end{aligned}
\end{equation}
\end{definition}

By Definition~\ref{def:absolute-continuity}, ``$m_1 \ll m_2$'' means that $m_1$ has at least as many measure-zero sets as $m_2$, and is therefore, in a sense, smaller.

Because we are doing probability, we need to deal only with nonnegative derivatives, for which operations have simpler closure conditions.
The types of nonnegative integration and differentiation are
\begin{equation}
\begin{aligned}
	int^+ &: (X \to [0,+\infty)) \tto (Set~X \pto [0,+\infty]) \tto (Set~X \pto [0,+\infty]) \\
	diff^+ &: (Set~X \pto [0,+\infty]) \tto (Set~X \pto [0,+\infty]) \tto (X \to [0,+\infty))
\end{aligned}
\label{eqn:int-diff-types}
\end{equation}
The function $diff^+$ returns a \keyword{Radon-Nikod\'ym derivative}.
Such derivatives are named after the following theorem, which gives circumstances under which $diff^+~m_1~m_2$ exists, and states that $int^+$ is the left inverse of $diff^+$ (with second arguments held constant).

\begin{lemma}[Radon-Nikod\'ym theorem]
\label{lem:radon-nikodym}
If $m_1,m_2 : Set~X \pto [0,+\infty]$ are $\sigma$-finite measures and $m_1 \ll m_2$, then $diff^+~m_1~m_2$ exists and $m_1 = int^+~(diff^+~m_1~m_2)~m_2$.
\end{lemma}

The function $diff^+~m_1~m_2 : X \to [0,+\infty)$ is often called the \keyword{density} of $m_1$ with respect to $m_2$.
(``With respect to $m_2$'' is usually not stated when $m_2$ is Lebesgue measure---i.e. length, area, or volume---as with the densities of most common probability distributions.)
By Lemma~\ref{lem:real-almost-everywhere-equality} (almost-everywhere equality of real functions), $diff^+~m_1~m_2$ is unique up to equality $m_2$-a.e.

By analogy to calculus, we should expect $diff^+$ to be the left inverse of $int^+$ (with second arguments held constant).
It is, up to equality $m_2$-a.e.
\begin{lemma}
\label{lem:diff-left-inverse-int}
If $f_1 : X \to [0,+\infty)$ is measurable and $m_2 : Set~X \pto [0,+\infty]$ is a $\sigma$-finite measure, then $int^+~f_1~m_2 \ll m_2$ and $f_1 = diff^+~(int^+~f_1~m_2)~m_2$ ($m_2$-a.e.).
\end{lemma}
\begin{comment}
\begin{proof}
\begin{align*}
	&int~(f_1 - diff^+~(int^+~f_1~m_2)~m_2)~m_2
\\
	&\tab=\ int~f_1~m_2 - int~(diff^+~(int^+~f_1~m_2)~m_2)~m_2
\\
	&\tab=\ int^+~f_1~m_2 - int^+~(diff^+~(int^+~f_1~m_2)~m_2)~m_2
\\
	&\tab=\ int^+~f_1~m_2 - int^+~f_1~m_2
\end{align*}
\end{proof}
\end{comment}

The previous two theorems are analogous to the two parts of the fundamental theorem of calculus.
The next two theorems are analogous to others in calculus: that Radon-Nikod\'ym differentiation is linear in its first argument.

\begin{lemma}
\label{lem:diff-distributes-over-addition}
Let $m_1,m_2,m : Set~X \pto [0,+\infty]$ be $\sigma$-finite measures with $m_1 \ll m$ and $m_2 \ll m$.
Then $m_1 + m_2 \ll m$ and $diff^+~(m_1 + m_2)~m\ =\ diff^+~m_1~m + diff^+~m_2~m$ ($m$-a.e.).
\end{lemma}

\begin{lemma}
\label{lem:diff-distributes-over-scaling}
Let $m_1,m_2 : Set~X \pto [0,+\infty]$ be $\sigma$-finite measures with $m_1 \ll m_2$.
For all $\alpha \ge 0$ and $\beta > 0$, $\alpha \cdot m_1 \ll \beta \cdot m_2$ and $diff^+~(\alpha \cdot m_1)~(\beta \cdot m_2)\ =\ \dfrac{\alpha}{\beta} \cdot diff^+~m_1~m_2$ ($m$-a.e.).
\end{lemma}

As in differentiation in calculus, there is a chain rule for $diff^+$:
\begin{lemma}
\label{lem:diff-chain-rule}
Let $m_1,m_2,m_3 : Set~X \pto [0,+\infty]$ be $\sigma$-finite measures with $m_1 \ll m_2$ and $m_2 \ll m_3$.
Then $m_1 \ll m_3$ and $diff^+~m_1~m_3\ =\ diff^+~m_1~m_2 \cdot diff^+~m_2~m_3$ ($m_3$-a.e.).
\end{lemma}

Now we finally get to rules for which there is no analogy in calculus.
First, a rule for reciprocals [XXX: probably don't need this---double-check]:
\begin{lemma}
\label{lem:diff-reciprocal-rule}
Let $m_1,m_2 : Set~X \pto [0,+\infty]$ be $\sigma$-finite measures with $m_2 \ll m_1$ and $m_1 \ll m_2$.
Then $diff^+~m_1~m_2\ =\ 1~{/}~diff^+~m_2~m_1$ ($m_1$-a.e. or $m_2$-a.e.).
\end{lemma}

Second, a way to integrate out densities, or to use differentiation to change the base measure in Lebesgue integration:
\begin{lemma}[change of measure]
\label{lem:diff-change-measure}
Let $m_1,m_2 : Set~X \pto [0,+\infty]$ be $\sigma$-finite measures with $m_1 \ll m_2$, and $g : X \to \Re$ be measurable.
Then $int~g~m_1\ =\ int~(g \cdot diff^+~m_1~m_2)~m_2$.
\end{lemma}

XXX: intro the following; something about composing two processes to yield another, where the second depends on the outcome of the first

\begin{definition}[transition kernel]
\label{def:transition-kernel}
A function $k : X \to Set~Y \pto [0,+\infty]$ is a \keyword{transition kernel} when both of the following hold.
\begin{itemize}
	\item For all $a \in X$, $k~a$ is a measure.
	\item For all $B \in \Sigma~Y$, $\fun{a \in X} k~a~B$ is measurable.
\end{itemize}
For any measure property $P$, we say $k$ has property $P$ when for all $a \in X$, $P~(k~a)$. [XXX: won't need this last statement if I go with uniformly $\sigma$-finite]
\end{definition}

\begin{definition}[uniformly $\sigma$-finite]
A transition kernel $k : X \to Set~Y \pto [0,+infty]$ is \keyword{uniformly $\sigma$-finite} when there exists an at-most-countable partition $s : \Nat \to Set~Y$ such that $k~a~(s~n) < +\infty$ for all $a \in X$ and $n \in \Nat$.
\end{definition}

\begin{lemma}[finite transition kernel products]
\label{lem:finite-transition-kernel-products}
Let $m : Set~X \to [0,+\infty]$ be a $\sigma$-finite measure and $k : X \to Set~Y \pto [0,+\infty]$ be a uniformly $\sigma$-finite transition kernel.
There exists a unique $\sigma$-finite measure $m \times k : Set~\pair{X,Y} \pto [0,+\infty]$ defined by extending $(m \times k)~(A \times B) = int^+~(\fun{a \in X} k~a~B)~m~A$ to a product measure.
\end{lemma}

\begin{lemma}[Fubini's for transition kernels]
\label{lem:fubini-for-transition-kernels}
Let $m : Set~X \to [0,+\infty]$ be a $\sigma$-finite measure and $k : X \to Set~Y \pto [0,+\infty]$ be a uniformly $\sigma$-finite transition kernel. If $f : X \times Y \to [-\infty,+\infty]$ is measurable, and nonnegative or $(m \times k)$-integrable, then
\begin{equation}
\begin{aligned}
	&int~f~(m \times k)~(X \times Y)
\\
	&\tab =\ 
	int~(\fun{a \in X} int~(\fun{b \in Y} f~\pair{a,b})~(k~a)~Y)~m~X
\end{aligned}
\end{equation}
\end{lemma}


\section{Sampling Proofs}
\label{sec:sampling-proofs}

\begin{theorem}[partitioned importance sampling correctness]
%\label{thm:partitioned-importance-sampling-correctness}
Let $X$, $P$, $N$, $s$, $p$, and $Q$ as in Definition~\ref{def:partitioned-importance-sampling} (partitioned importance sampling) such that $subcond~P~(s~n) \ll Q~n$ for all $n \in N$. Define $P_N : Set~N \to [0,1]$ by extending $p$ to a measure.

If $g : X \to \Re$ is a $P$-integrable mapping, and
\begin{equation}
\begin{aligned}
	&g' : N \times X \to \Re \\
	&g'~\pair{n,a}\ :=\ g~a \cdot \dfrac{1}{p~n} \cdot diff^+~(subcond~P~(s~n))~(Q~n)~a
\end{aligned}
\end{equation}
then $int~g'~(P_N \times Q)~(N \times X)\ =\ int~g~P~X$.
\end{theorem}
\begin{proof}
Expand $g'$ in the left-hand side of the equality, apply Lemma~\ref{lem:fubini-for-transition-kernels} (Fubini's for transition kernels), rearrange inside terms, and apply Lemma~\ref{lem:diff-change-measure} (change of measure) to cancel $Q~n$:
\begin{displaybreaks}
\begin{align*}
	&int~g'~(P_N \times Q)~(N \times X)
\\*
	&\tab =\ int~(\fun{\pair{n,a} \in N \times X} g~a \cdot \dfrac{1}{p~n} \cdot diff^+~(subcond~P~(s~n))~(Q~n)~a)~(P_N \times Q)~(N \times X)
\\
	&\tab =\ int~(\fun{n \in N} int~(\fun{a \in X} g~a \cdot \dfrac{1}{p~n} \cdot diff^+~(subcond~P~(s~n))~(Q~n)~a)~(Q~n)~X)~P_N~N
\\
	&\tab =\ int~(\fun{n \in N} int~(g \cdot \dfrac{1}{p~n} \cdot diff^+~(subcond~P~(s~n))~(Q~n))~(Q~n)~X)~P_N~N
\\
	&\tab =\ int~(\fun{n \in N} int~(g \cdot \dfrac{1}{p~n})~(subcond~P~(s~n))~X)~P_N~N
\intertext{Because $N$ is at most countable, turn integration into summation; then use $\sigma$-additivity of measures:}
	&\tab =\ \sum_{n \in N} p~n \cdot int~(g \cdot \dfrac{1}{p~n})~(subcond~P~(s~n))~X
\\
	&\tab =\ \sum_{n \in N} int~g~(subcond~P~(s~n))~X
\\
	&\tab =\ \sum_{n \in N} int~g~P~(s~n)
\\
	&\tab =\ int~g~P~(\U_{n \in N}~(s~n))
\\*
	&\tab =\ int~g~P~X
\\[-2.25\baselineskip]
\end{align*}
\end{displaybreaks}
\end{proof}


\begin{theorem}
Let $m_1,m_2 : Set~X \pto [0,+\infty]$ be $\sigma$-finite measures such that $m_1 \ll m_2$, and $k : X \to Set~Y \pto [0,+\infty]$ be a uniformly $\sigma$-finite transition kernel.
Then $m_1 \times k \ll m_2 \times k$ and $diff^+~(m_1 \times k)~(m_2 \times k)\ =\ (\fun{\pair{a,b} \in X \times Y} diff^+~m_1~m_2~a)$ ($m_2 \times k$-a.e.).
\end{theorem}
\begin{proof}
XXX: prove absolute continuity

By Lemma~\ref{lem:real-almost-everywhere-equality}, they are equal $m_2 \times k$-a.e. if and only if their indefinite integrals w.r.t. $m_2 \times k$ are equal.

Let $A$ and $B$ be measurable subsets of $X$ and $Y$ respectively.
Starting from the left-hand side, apply Lemma~\ref{lem:diff-change-measure} (change of measure):
\begin{equation}
\begin{aligned}
	&int~(diff^+~(m_1 \times k)~(m_2 \times k))~(m_2 \times k)~(A \times B)
\\
	&\tab=\ int~(\fun{\pair{a,b} \in X \times Y} 1)~(m_1 \times k)~(A \times B)
\\
	&\tab=\ (m_1 \times k)~(A \times B)
\end{aligned}
\end{equation}
From the right-hand side, apply Lemma~\ref{lem:fubini-for-transition-kernels} (Fubini's for transition kernels), rearrange terms, apply Lemma~\ref{lem:diff-change-measure} (change of measure), and Lemma~\ref{lem:finite-transition-kernel-products} (finite transition kernel products):
\begin{equation}
\begin{aligned}
	&int~(\fun{\pair{a,b} \in X \times Y} diff^+~m_1~m_2~a)~(m_2 \times k)~(A \times B)
\\
	&\tab=\ int~(\fun{a \in X} int~(\fun{b \in Y} diff^+~m_1~m_2~a)~(k~a)~B)~m_2~A
\\
	&\tab=\ int~(diff^+~m_1~m_2 \cdot \fun{a \in X} int~(\fun{b \in Y} 1)~(k~a)~B)~m_2~A
\\
	&\tab=\ int~(\fun{a \in X} k~a~B)~m_1~A
\\
	&\tab=\ (m_1 \times k)~(A \times B)
\end{aligned}
\end{equation}
By uniqueness of $\sigma$-finite product measures, the integrals are equal.
\end{proof}

It is not hard to extend the preceeding theorem to arbitrary sublists of finite lists, or to arbitrary finite substructures of any algebraic data type, by finite induction.
But we need a version of it for arbitrary finite substructures of infinite binary trees, which we have defined non-inductively as mappings $R := J \to [0,1]$ from tree indexes to reals.

The main idea is to define a transformation from any $r \in R$ to a pair $\pair{r_{fin},r_{inf}}$, where $r_{fin}$ is a finite substructure and $r_{inf}$ is the rest of it, and apply Theorem~[XXX: preceeding].
The proof is easier to do abstractly: without specifying the structure of $r$, without requiring the substructure to be finite, and without requiring the transformation to be a projection.

\begin{theorem}
Let $m_1,m_2 : Set~X \pto [0,+\infty]$ be $\sigma$-finite measures such that $m_1 \ll m_2$, and $k : X \to Set~Y \pto [0,+\infty]$ be a uniformly $\sigma$-finite transition kernel.
Let $f : \Omega \to X \times Y$ be measurable and invertible, and let $\mu_1,\mu_2 : Set~\Omega \pto [0,1]$ so that
\begin{equation}
\begin{aligned}
	&\mu_1~\Omega'\ =\ (m_1 \times k)~(image~f~\Omega') \\
	&\mu_2~\Omega'\ =\ (m_2 \times k)~(image~f~\Omega')
\end{aligned}
\end{equation}
Then $diff^+~\mu_1~\mu_2\ =\ \fun{\omega \in \Omega} diff^+~m_1~m_2~(fst~(f~\omega))$ ($\mu_2$-a.e.).
\end{theorem}
\begin{proof}
Suppose $\Omega' \subseteq \Omega$ is measurable and let $C := image~f~\Omega'$.
Let $f^{-1}$ be the inverse of $f$.
[XXX: justify steps]
\begin{displaybreaks}
\begin{align*}
	&int~(diff^+~\mu_1~\mu_2)~\mu_2~\Omega'
\\*
	&\tab=\ int~(1_{\Omega'} \cdot diff^+~\mu_1~\mu_2)~\mu_2~\Omega
\\
	&\tab=\ int~1_{\Omega'}~\mu_1~\Omega
\\
	&\tab=\ int~(1_{\Omega'} \circ\map f^{-1})~(m_1 \times k)~(X \times Y)
\\
	&\tab=\ int~1_C~(m_1 \times k)~(X \times Y)
\\
	&\tab=\ int~(1_C \cdot diff^+~(m_1 \times k)~(m_2 \times k))~(m_2 \times k)~(X \times Y)
\\
	&\tab=\ int~(1_C \cdot \fun{\pair{a,b} \in X \times Y} diff^+~m_1~m_2~a)~(m_2 \times k)~(X \times Y)
\\
	&\tab=\ int~((1_C \cdot \fun{\pair{a,b} \in X \times Y} diff^+~m_1~m_2~a) \circ\map f)~\mu_2~\Omega
\\
	&\tab=\ int~(1_{\Omega'} \cdot (\fun{\pair{a,b} \in X \times Y} diff^+~m_1~m_2~a) \circ\map f)~\mu_2~\Omega
\\
	&\tab=\ int~((\fun{\pair{a,b} \in X \times Y} diff^+~m_1~m_2~a) \circ\map f)~\mu_2~\Omega'
\\
	&\tab=\ int~(\fun{\omega \in \Omega} diff^+~m_1~m_2~(fst~(f~\omega)))~\mu_2~\Omega'
\end{align*}
\end{displaybreaks}
Apply Lemma~\ref{lem:real-almost-everywhere-equality} (real function a.e. equality).
\end{proof}

Thus, two measures $\mu_1$ and $\mu_2$ on infinite structures that can be decomposed into products $m_1 \times k$ and $m_2 \times k$ such that $diff^+~m_1~m_2$ exists have a Radon-Nikod\'ym derivative that can be completely characterized in terms of $diff^+~m_1~m_2$.

Application to infinite binary trees merely requires defining the transformation $f$, which is clearly invertible.

\begin{corollary}
Let $J' \subseteq J$, let $m_1,m_2 : Set~(J' \to [0,1]) \pto [0,1]$ be $\sigma$-finite measures such that $m_1 \ll m_2$, and let $k : (J' \to [0,1]) \to (Set~(J \w J' \to [0,1]) \pto [0,1])$ be a uniformly $\sigma$-finite transition kernel.
Let
\begin{equation}
\begin{aligned}
	&f : R \to (J' \to [0,1]) \times (J \w J' \to [0,1]) \\
	&f~r\ :=\ \pair{restrict~r~J',restrict~r~(J \w J')}
\end{aligned}
\end{equation}
and let $\mu_1,\mu_2 : Set~R \pto [0,1]$ so that
\begin{equation}
\begin{aligned}
	&\mu_1~R'\ =\ (m_1 \times k)~(image~f~R') \\
	&\mu_2~R'\ =\ (m_2 \times k)~(image~f~R')
\end{aligned}
\end{equation}
Then $diff^+~\mu_1~\mu_2\ =\ \fun{r \in R} diff^+~m_1~m_2~(restrict~r~J')$ ($\mu_2$-a.e.).
\end{corollary}


\mathversion{normal}
