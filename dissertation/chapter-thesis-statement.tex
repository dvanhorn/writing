\begin{quote}
\textit{This branch of mathematics [Probability] is the only one, I believe, in which good writers frequently get results which are entirely erroneous.}

\hfill Charles S. Peirce
\end{quote}

Probability is notorious for being stubbornly counterintuitive.
Any automation of probabilistic calculations or reasoning is therefore helpful.
In Bayesian statistics, automation is taking the form of probabilistic languages for specifying random processes, which compute answers to questions about them under constraints.

Such languages should be made to meet a mathematical specification.
The reason is simple: if a probabilistic language is made to always meet its maker's expectations, it is almost certainly faulty.

XXX: other reasons

XXX: short segue to...

\begin{quote}
\textbf{Functional programming theory and measure-theoretic probability provide a solid foundation for trustworthy, useful languages for constructive probabilistic modeling and inference.}
\end{quote}

\section{Statement Terms}

\paragraph{Funcional Programming Theory.} Blah blah blah.

Blah.

\paragraph{Measure-Theoretic Probability.} Blah blah blah.

Blah.

\paragraph{Trustworthy.} Blah blah blah.

Blah.

\paragraph{Useful.} Blah blah blah.

Blah.

\paragraph{Constructive Probabilitic Modeling.} Blah blah blah.

Blah.

\paragraph{Probabilistic Inference.} Blah blah blah.

Blah.

\section{Proof and Supporting Evidence}

\paragraph{Semantics.} Blah blah blah.

Blah.

\paragraph{Properties.} Blah blah blah.

Blah.

\paragraph{Test Cases.} Blah blah blah.

Blah.
