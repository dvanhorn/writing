\mathversion{sans}

Proving measurability is critical in proving correctness, in that it establishes that the outputs of all programs have sensible distributions.
While critical, it is somewhat distracting to the main narrative.
Instead of ignoring measurability, however, as is so often done, we have moved it near the end, where readers that are somehow \emph{still starving for even more mathematics} can devour it and---possibly---finally be satiated.

\section{Basic Definitions}

We assume readers are familiar with topology or measure theory.
For readers familiar with topology but not measure theory, we review the necessary fundamentals by analogy to topology.

The analogue of a topology of open sets is a $\sigma$-algebra of measurable sets.

\begin{definition}[$\sigma$-algebra, measurable set]
A collection of sets $\A \subseteq \powerset~X$ is called a \keyword{$\sigma$-algebra} on $X$ if it contains $X$ and is closed under complements and countable unions.
The sets in $\A$ are called \keyword{measurable sets}.
\end{definition}

$X \w X = \emptyset$, so $\emptyset \in \A$.
Additionally, it follows from De Morgan's law that $\A$ is closed under countable intersections.

The analogue of continuity is measurability.

\begin{definition}[measurable mapping]
Let $\A$ and $\B$ be $\sigma$-algebras on $X$ and $Y$.
A mapping $g : X \pto Y$ is $\A!\B$-\keyword{measurable} if for all $B \in \B$, $preimage~g~B \in \A$.
\end{definition}

When the domain and codomain $\sigma$-algebras $\A$ and $\B$ are clear from context, we will simply say $g$ is \keyword{measurable}.

Measurability is usually a weaker condition than continuity.
For example, with respect to the $\sigma$-algebra generated from $\Re$'s standard topology (i.e. using the standard topology as a sort of ``base''), measurable $\Re \pto \Re$ functions may have countably many discontinuities.
Likewise, real comparison functions such as $(=)$, $(<)$, $(>)$ and their negations are measurable, but not continuous.

Product $\sigma$-algebras are defined analogously to product topologies.

\begin{definition}[finite product $\sigma$-algebra]
\label{def:finite-product-sigma-algebra}
Let $\A_1$ and $\A_2$ be $\sigma$-algebras on $X_1$ and $X_2$, and define $X := X_1 \times X_2$.
The \keyword{product $\sigma$-algebra} $\A_1 \otimes \A_2$ is the smallest (i.e. coarsest) $\sigma$-algebra for which $mapping~fst~X$ and $mapping~snd~X$ are measurable.%
\end{definition}

\begin{definition}[arbitrary product $\sigma$-algebra]
\label{def:arbitrary-product-sigma-algebra}
Let $\A$ be a $\sigma$-algebra on $X$.
The \keyword{product $\sigma$-algebra} $\A^{\otimes J}$ is the smallest $\sigma$-algebra for which, for all $j \in J$, $mapping~(\pi~j)~(J \to X)$ is measurable.%
\end{definition}

\section{Measurable Pure Computations}

It is easier to prove measurability of pure computations than to prove measurability of partial, probabilistic ones.
Further, we can use the results to prove that the interpretations of all partial, probabilistic expressions are measurable.

We must first define what it means for a \emph{computation} to be measurable.

\begin{definition}[measurable mapping arrow computation]
\label{def:measurable-mapping-arrow-computation}
Let $\A$ and $\B$ be $\sigma$-algebras on $X$ and $Y$.
A computation $g : X \mapto Y$ is $\A!\B$-\keyword{measurable} if $g~A^*$ is an $\A!\B$-measurable mapping, where $A^*$ is $g$'s maximal domain.%
\end{definition}

\begin{theorem}[maximal domain measurability]
Let $g : X \mapto Y$ be an $\A!\B$-measurable mapping arrow computation.
Its maximal domain $A^*$ is in $\A$.
\end{theorem}
\begin{proof}
Because $g~A^*$ is measurable, $preimage~(g~A^*)~Y = A^*$ is in $\A$.
\end{proof}

Mapping arrow computations can be applied to sets other than their maximal domains.
We need to ensure doing so yields a measurable mapping, at least for measurable subsets of $A^*$.
Fortunately, that is true without any extra conditions.

\begin{lemma}
\label{lem:restricted-mappings-are-measurable}
Let $g : X \pto Y$ be an $\A!\B$-measurable mapping.
For any $A \in \A$, $restrict~g~A$ is $\A!\B$-measurable.%
\end{lemma}

\begin{theorem}
\label{thm:restricted-computations-are-measurable}
Let $g : X \mapto Y$ be an $\A!\B$-measurable mapping arrow computation with maximal domain $A^*$.
For all $A \subseteq A^*$ with $A \in \A$, $g~A$ is $\A!\B$-measurable.
\end{theorem}
\begin{proof}
By Theorem~\ref{thm:mapping-arrow-restriction} (mapping arrow restriction) and Lemma~\ref{lem:restricted-mappings-are-measurable}.
\end{proof}

We do not need to prove all interpretations using $\meaningof{\cdot}\gen$ are measurable.
However, we do need to prove mapping arrow combinators preserve measurability.

\paragraph{Composition}
Proving compositions are measurable takes the most work.
The main complication is that, under measurable mappings, while \emph{preimages} of measurable sets are measurable, \emph{images} of measurable sets may not be.
We need the following four extra theorems to get around this.

\begin{lemma}[images of preimages]
\label{lem:images-of-preimages}
If $g : X \pto Y$ and $B \subseteq Y$, $image~g~(preimage~g~B) \subseteq B$.%
\end{lemma}

\begin{lemma}[expanded post-composition]
\label{lem:composition-expansion}
Let $g_1 : X \pto Y$ and $g_2 : Y \pto Z$ such that $range~g_1 \subseteq domain~g_2$, and let $g_2' : Y \pto Z$ such that $g_2 \subseteq g_2'$.
Then $g_2 \circ\map g_1 = g_2' \circ\map g_1$.%
\end{lemma}

\begin{theorem}[mapping arrow monotonicity]
\label{thm:mapping-arrow-monotonicity}
Let $g : X \mapto Y$.
For any $A' \subseteq A \subseteq A^*$, $g~A' \subseteq g~A$.%
\end{theorem}
\begin{proof}
By Theorem~\ref{thm:mapping-arrow-restriction} (mapping arrow restriction).
\end{proof}

\begin{theorem}[maximal domain subsets]
\label{thm:maximal-domain-subsets}
Let $g : X \mapto Y$. For any $A \subseteq A^*$, $domain~(g~A) = A$.%
\end{theorem}
\begin{proof}
Follows from Theorem~\ref{thm:domain-closure-operators}.
\end{proof}

Now we can prove measurability.

\begin{lemma}[$(\circ\map)$ measurability]
\label{lem:compositions-are-measurable}
If $g_1 : X \pto Y$ is $\A!\B$-measurable and $g_2 : Y \pto Z$ is $\B!\C$-measurable, then $g_2 \circ\map g_1$ is $\A!\C$-measurable.%
\end{lemma}

\begin{theorem}[$(\compmap)$ measurability]
If $g_1 : X \mapto Y$ is $\A!\B$-measurable and $g_2 : Y \mapto Z$ is $\B!\C$-measurable, then $g_1~\compmap~g_2$ is $\A!\C$-measurable.
\end{theorem}
\begin{proof}
Let $A^* \in \A$ and $B^* \in \B$ be respectively $g_1$'s and $g_2$'s maximal domains.
The maximal domain of $g_1~\compmap~g_2$ is $A^{**} := preimage~(g_1~A^*)~B^*$, which is in $\A$.
By definition,
\begin{equation}
	(g_1~\compmap~g_2)~A^{**} \ = \ 
		\lzfclet{
			g_1' & g_1~A^{**} \\
			g_2' & g_2~(range~g_1')
		}{g_2' \circ\map g_1'}
\end{equation}
By Theorem~\ref{thm:restricted-computations-are-measurable}, $g_1'$ is an $\A!\B$-measurable mapping.
Unfortunately, $g_2'$ may not be $\B!\C$-measurable when $range~g_1' \not\in \B$.

Let $g_2'' := g_2~B^*$, which is a $\B!\C$-measurable mapping.
By Lemma~\ref{lem:compositions-are-measurable}, $g_2'' \circ\map g_1'$ is $\A!\C$-measurable.
We need only show that $g_2' \circ\map g_1' = g_2'' \circ\map g_1'$, which by Lemma~\ref{lem:composition-expansion} is true if $range~g_1' \subseteq domain~g_2'$ and $g_2' \subseteq g_2''$.

By Theorem~\ref{thm:maximal-domain-subsets}, $A^{**} \subseteq A^*$ implies $domain~g_1' = A^{**}$.
By Theorem~\ref{thm:mapping-arrow-monotonicity} and Lemma~\ref{lem:images-of-preimages},
\begin{align*}
\numberthis
	range~g_1'
		%&\ =\ image~g_1'~A^{**} \\
		&\ =\ image~(g_1~A^{**})~(preimage~(g_1~A^*)~B^*) \\
		&\ =\ image~(g_1~A^*)~(preimage~(g_1~A^*)~B^*) \\
		&\ \subseteq\ B^*
\end{align*}
$range~g_1' \subseteq B^*$ implies (by Theorem~\ref{thm:maximal-domain-subsets}) that $domain~g_2' = range~g_1'$, and (by Theorem~\ref{thm:mapping-arrow-monotonicity}) that $g_2' \subseteq g_2''$.
\end{proof}

\paragraph{Pairing}
Proving pairing preserves measurability is straightforward given a corresponding theorem about mappings.

\begin{lemma}[$\pair{\cdot,\cdot}\map$ measurability]
\label{lem:pairings-are-measurable}
If $g_1 : X \pto Y_1$ is $\A!\B_1$-measurable and $g_2 : X \pto Y_2$ is $\A!\B_2$-measurable, then $\pair{g_1,g_2}\map$ is $\A!(\B_1 \otimes \B_2)$-measurable.%
\end{lemma}

\begin{theorem}[$(\pairmap)$ measurability]
If $g_1 : X \mapto Y_1$ is $\A!\B_1$-measurable and $g_2 : X \mapto Y_2$ is $\A!\B_2$-measurable, then $g_1~\pairmap~g_2$ is $\A!(\B_1 \otimes \B_2)$-measurable.
\end{theorem}
\begin{proof}
Let $A_1^*$ and $A_2^*$ be respectively $g_1$'s and $g_2$'s maximal domains.
The maximal domain of $g_1~\pairmap~g_2$ is $A^{**} := A_1^* \i A_2^*$, which is in $\A$.
By definition, $(g_1~\pairmap~g_2)~A^{**} = \pair{g_1~A^{**},g_2~A^{**}}\map$, which by Lemma~\ref{lem:pairings-are-measurable} is $\A!(\B_1 \otimes \B_2)$-measurable.
\end{proof}

\paragraph{Conditional}
Conditionals can be proved measurable given a theorem that ensures the measurability of \emph{finite} unions of disjoint, measurable mappings.
We will need the corresponding theorem for \emph{countable} unions further on, however.

\begin{lemma}[union of measurable mappings]
\label{lem:union-of-measurable-mappings}
The union of a countable set of $\A!\B$-measurable mappings with disjoint domains is $\A!\B$-measurable.%
\end{lemma}

\begin{theorem}[$\ifmap$ measurability]
If $g_1 : X \mapto Bool$, and $g_2 : X \mapto Y$ and $g_3 : X \mapto Y$ are respectively $\A!(\powerset~Bool)$-measurable and $\A!\B$-measurable, then $\ifmap~g_1~g_2~g_3$ is $\A!\B$-measurable.
\end{theorem}
\begin{proof}
Let $\A_1^*$, $\A_2^*$ and $\A_3^*$ be $g_1$'s, $g_2$'s and $g_3$'s maximal domains.
The maximal domain of $\ifmap~g_1~g_2~g_3$ is $A^{**}$, defined by
\begin{equation}
\begin{aligned}
	A_2^{**} &\ :=\ A_2^* \i preimage~(g_1~\A_1^*)~\set{true} \\
	A_3^{**} &\ :=\ A_3^* \i preimage~(g_1~\A_1^*)~\set{false} \\
	A^{**} &\ :=\ A_2^{**} \uplus A_3^{**}
\end{aligned}
\end{equation}
Because $preimage~(g_1~\A_1^*)~B \in \A$ for any $B \subseteq Bool$, $A^{**} \in \A$.
By definition,
\begin{equation}
	\ifmap~g_1~g_2~g_3~A^{**} \ = \ 
		\lzfclet{
			g_1' & g_1~A^{**} \\
			g_2' & g_2~(preimage~g_1'~\set{true}) \\
			g_3' & g_3~(preimage~g_1'~\set{false})
		}{g_2' \uplus\map g_3'}
\end{equation}
By hypothesis, $g_1'$, $g_2'$ and $g_3'$ are measurable mappings.
By Theorem~\ref{thm:mapping-arrow-restriction} (mapping arrow restriction), $g_2'$ and $g_3'$ have disjoint domains.
Apply Lemma~\ref{lem:union-of-measurable-mappings}.
\end{proof}

\paragraph{Laziness}
We must first prove measurability of an often-ignored corner case.

\begin{theorem}[measurability of $\emptyset$]
\label{thm:empty-mapping-measurable}
For any $\sigma$-algebras $\A$ and $\B$, the empty mapping $\emptyset$ is $\A!\B$-measurable.%
\end{theorem}
\begin{proof}
For any $B \in \B$, $preimage~\emptyset~B = \emptyset$, and $\emptyset \in \A$.
\end{proof}

\begin{theorem}[measurability under $\lazymap$]
Let $g : 1 \tto (X \mapto Y)$. If $g~0$ is $\A!\B$-measurable, then $\lazymap~g$ is $\A!\B$-measurable.
\end{theorem}
\begin{proof}
The maximal domain $A^{**}$ of $\lazymap~g$ is that of $g~0$.
By definition,
\begin{equation}
	\lazymap~g~A^{**} \ = \ if~(A^{**} = \emptyset)~\emptyset~(g~0~A^{**})
\end{equation}
If $A^{**} = \emptyset$, then $\lazymap~g~A^{**} = \emptyset$; apply Theorem~\ref{thm:empty-mapping-measurable}.
If $A^{**} \neq \emptyset$, then $\lazymap~g = g~0$, which is $\A!\B$-measurable.
\end{proof}

\section{Measurable Probabilistic Computations}

As with pure computations, we must first define what it means for an effectful computation to be measurable.

\begin{definition}[measurable mapping* arrow computation]
Let $\A$ and $\B$ be $\sigma$-algebras on $(R \times T) \times X$ and $Y$.
A computation $g : X \pmapto Y$ is $\A!\B$-\keyword{measurable} if $g~j_0$ is an $\A!\B$-measurable mapping arrow computation.
\end{definition}

\begin{theorem}
If $g : X \pmapto Y$ is $\A!\B$-measurable, then for all $j \in J$, $g~j$ is an $\A!\B$-measurable mapping arrow computation.
\end{theorem}
\begin{proof}
By induction on $J$: if $g~j$ is measurable, so are $g~(left~j)$ and $g~(right~j)$.
\end{proof}

To make general measurability statements about computations, whether they have flat or product types, it helps to have a notion of a standard $\sigma$-algebra.

\begin{definition}[standard $\sigma$-algebra]
\label{def:standard-sigma-algebra}
For a set $X$ used as a type, $\Sigma~X$ denotes its \mykeyword{standard $\sigma$-algebra}, which must be defined under the following constraints:
\begin{align}
	\Sigma~\pair{X_1,X_2} & = \Sigma~X_1 \otimes \Sigma~X_2
	\label{eqn:standard-finite-product-rule}
\\
	\Sigma~(J \to X) & = (\Sigma~X)^{\otimes J}
	\label{eqn:standard-arbitrary-product-rule}
%\\
%	X' \in \Sigma~X \ \implies \ \Sigma~X' & = \setb{X' \i A}{A \in \Sigma~X}
%	\label{eqn:standard-subset-rule}
\end{align}
\end{definition}

From here on, when no $\sigma$-algebras are given, ``measurable'' means ``measurable with respect to standard $\sigma$-algebras.''

The following definitions allow distinguishing the results of conditional expressions and any two branch traces:
\begin{align}
	\Sigma~Bool &\ ::=\ \powerset~Bool
	\label{eqn:standard-boolean-rule}
\\
	\Sigma~T &\ ::=\ \powerset~T
	\label{eqn:standard-traces-rule}
\end{align}

\begin{lemma}[measurable mapping arrow lifts]
$\arrmap~id$, $\arrmap~fst$ and $\arrmap~snd$ are measurable.
$\arrmap~(const~b)$ is measurable if $\set{b}$ is a measurable set.
For all $j \in J$, $\arrmap~(\pi~j)$ is measurable.
\end{lemma}

%XXX: should that really be a lemma? $\arrmap$ removes $\bot$ from the domain

\begin{corollary}[measurable mapping* arrow lifts]
$\arrpmap~id$, $\arrpmap~fst$ and $\arrpmap~snd$ are measurable.
$\arrpmap~(const~b)$ is measurable if $\set{b}$ is a measurable set.
$random\pmap$ and $branch\pmap$ are measurable.
\end{corollary}

\begin{theorem}[$AStore$ combinators preserve measurability]
\label{thm:astore-measurability-transfer}
Every $AStore$ arrow combinator produces measurable mapping* computations from measurable mapping* computations.%
\end{theorem}
\begin{proof}
$AStore$'s combinators are defined in terms of the base arrow's combinators and $\arrmap~fst$ and $\arrmap~snd$.
\end{proof}

\begin{theorem}[$\convifpmap$ measurability]
$\convifpmap$ is measurable.
\end{theorem}
\begin{proof}
$branch\pmap$ is measurable, and $\arrmap~agrees$ is measurable by~\eqref{eqn:standard-boolean-rule}.
\end{proof}

We can now prove all nonrecursive programs measurable by induction.

\begin{definition}[finite expression]
\label{def:finite-expression}
A \mykeyword{finite expression} is any expression for which no subexpression is a first-order application.%
\end{definition}

\begin{theorem}[all finite expressions are measurable]
\label{thm:nonrecursive-programs-are-measurable}
For all finite expressions $\mathit{e}$, $\meaningof{\mathit{e}}\pmap$ is measurable.%
\end{theorem}
\begin{proof}
By structural induction and the above theorems.
\end{proof}

Now all we need to do is represent recursive programs as a net of finite expressions, and take a sort of limit.

\begin{theorem}[approximation with finite expressions]
\label{thm:nonrecursive-approximation}
Let $g := \meaningofconv{\mathit{e}}\pmap : X \pmapto Y$ and $t \in T$.
Define $A := (R \times \set{t}) \times X$.
There is a finite expression $\mathit{e'}$ for which $\meaningof{\mathit{e'}}\pmap~j_0~A = g~j_0~A$.%
\end{theorem}
\begin{proof}
Let the index prefix $J'$ contain every $j$ for which $t~j \neq \bot$.
To construct $\mathit{e'}$, exhaustively apply first-order functions in $\mathit{e}$, but replace any $\convifpmap$ whose index $j$ is not in $J'$ with the equivalent expression $\bot$.
Because $\mathit{e}$ is well-defined, recurrences must be guarded by $if$, so this process terminates after finitely many first-order applications.
\end{proof}

\begin{theorem}[all probabilistic expressions are measurable]
\label{thm:everything-is-measurable}
For all expressions $\mathit{e}$, $\meaningofconv{\mathit{e}}\pmap$ is measurable.%
\end{theorem}
\begin{proof}
Let $g := \meaningofconv{\mathit{e}}\pmap$ and $g' := g~j_0~((R \times T) \times X)$.
By Corollary~\ref{cor:correct-convergence} (correct computation everywhere), $g' = g~j_0~A^*$ where $A^*$ is $g$'s maximal domain; thus we need only show $g'$ is a measurable mapping.

By Theorem~\ref{thm:mapping-arrow-restriction} (mapping arrow restriction),
\begin{equation}
	g' \ =\ \bigcup\limits_{t \in T} g~j_0~((R \times \set{t}) \times X)
\end{equation}
By Theorem~\ref{thm:nonrecursive-approximation} (approximation with finite expressions), for every $t \in T$, there is a finite expression whose interpretation agrees with $g$ on $(R \times \set{t}) \times X$.
Therefore, by Theorem~\ref{thm:nonrecursive-programs-are-measurable} (all finite expressions are measurable), $g~j_0~((R \times \set{t}) \times X)$ is a measurable mapping.
By Theorem~\ref{thm:mapping-arrow-restriction} (mapping arrow restriction), they have disjoint domains.
By Lemma~\ref{lem:union-of-measurable-mappings} (union of measurable mappings), their union is measurable.
\end{proof}

Theorem~\ref{thm:everything-is-measurable} remains true when $\meaningof{\cdot}\gen$ is extended with any rule whose right side is measurable, including rules for real arithmetic, equality, inequality and limits.
More generally, any continuous or (countably) piecewise continuous function can be made available as a language primitive, as long as its domain's and codomain's standard $\sigma$-algebras are generated from their topologies.

It is not difficult to compose $\meaningof{\cdot}\gen$ with another semantic function that defunctionalizes lambda expressions.
Thus, the interpretations of all expressions in higher-order languages are measurable.

\section{Measurable Projections}

If $g := \meaningofconv{\mathit{e}}\pmap : X \pmapto Y$, the probability of a measurable output set $B \in \Sigma~Y$ is
\begin{equation}
	P~(image~(fst~\arrowcomp~fst)~(preimage~(g~j_0~A^*)~B))
\end{equation}
Unfortunately, projections are generally not measurable.
Fortunately, for interpretations of programs $\meaningofconv{\mathit{p}}\pmap$, for which $X = \set{\pair{}}$, we have a special case.

\begin{theorem}[measurable finite projections]
\label{thm:measurable-projections}
Let $A \in \Sigma~\pair{X_1,X_2}$.
If $X_2$ is at most countable and $\Sigma~X_2 = \powerset~X_2$, then $image~fst~A \in \A_1$.%
\end{theorem}
\begin{proof}
Because $\Sigma~X_2 = \powerset~X_2$, $A$ is a countable union of rectangles of the form $A_1 \times \set{a_2}$, where $A_1 \in \Sigma~X_1$ and $a_2 \in X_2$.
Because $image~fst$ distributes over unions, $image~fst~A$ is a countable union of sets in $\Sigma~X_1$.
\end{proof}

\begin{theorem}
Let $g : X \pmapto Y$ be measurable.
If $X$ is at most countable and $\Sigma~X = \powerset~X$, for all $B \in \Sigma~Y$, $image~(fst~\arrowcomp~fst)~(preimage~(g~j_0~A^*)~B) \in \Sigma~R$.
\end{theorem}
\begin{proof}
$T$ is countable and $\Sigma~T = \powerset~T$ by~\eqref{eqn:standard-traces-rule}; apply Theorem~\ref{thm:measurable-projections} twice.
\end{proof}

In particular, for $\meaningofconv{\mathit{p}}\pmap : \set{\pair{}} \pmapto Y$, the probabilities of $\Sigma~Y$ are well-defined.

\mathversion{normal}
